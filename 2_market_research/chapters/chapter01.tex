\subsection{Stakeholder Requirements}
Based on the requirements of the stakeholder, the platform needs to be able to,

\begin{itemize}
    \item navigate the user to an exhibit or room through the use of augmented reality.
    \item to display navigational routes in real time.
    \item calculate the shortest route to the user specified location. 
    \item work transferably in other museums/commercial spaces.
    \item contain accessibility features such as magnified text and inverted colours for example.
\end{itemize}

\subsection{Museum Visitors Responses}
After consulting and meeting with our stakeholders and potential users of the proposed application, we were able to gather a better understanding of what the apparent need was, in the relative market regarding commercial spaces and museums in particular. Out of the 21 responses we received, 15 potential users had admitted to visiting museums at least once a month. This shows that there is some level of frequency in their visits and that there is something that can be offered to this group of people. As our proposed application intends for the use of navigation around museums and commercial spaces, an important question that had to be asked was "do you find yourself using the maps in the museum more than once?" - a very reassuring 100\% of visitors had agreed that they did in fact refer to the maps around the museum more than once, some respondents going on to say that they referred to it over 10 times. However, these maps are not free. In most museums, including the Natural History Museum and the Science Museum in London, require a fee of £1 in order to have access to the paper maps they have available. This shows that there is an evident need for an accessible tool other than the maps around the museum in order to assist visitors' navigation around the museum. In addition to this, 18 of the respondents had agreed they would much rather prefer using their phone as a means of navigation rather than the paper maps that are currently available to assist in their navigation around the museum. These responses that we received first-hand were very reassuring for us as developers, as it brings to light an evident need for these visitors to have access to an improved navigation solution.

\subsection{Museum Staff Responses}
One key stakeholder whom would be affected are the museum staff as they would be instrumental to any on-the-ground assistance and would be the first port of help besides that of any virtual help featuring in within the project. Furthermore, the platform should endeavour to make the jobs undertaken by museum staff with relation to the parameters of platform easier.  
The chief stakeholder requirements, with relation to museum staff are to,
\begin{itemize}
    \item Exhibit an effective and easy-to-use design. 
    \item Be economic and effective in its use of data, given that most data the app would download would be sourced from in-museum Wi-Fi. 
    \item Written content and other media to be within control of the museum.
\end{itemize}
When meeting staff at the Natural History Museum and the Science Museum, while it was our intention the gauge requirements from a range of museum staff, it became apparent to target staff who would regularly come into contact with museum-goers would be our chief method of distribution and a first point of human contact for the users. During our field research we spoke to museum-floor staff and receptionists. Most importantly, all staff members whom we spoke to had received a navigational inquiry be it from themselves or members of the public. All staff responded positively to the use of a phone but citied concerns about efficiency of the concept with specific references to battery performance and data usage. One major concern highlighted by every member of museum staff was ease of use, and therefore, we suggested to supply a solution to this by designing a simple and efficient graphical user interface. Another issue we agreed on was the problem that became more apparent upon speaking to staff than before was the mechanism were the project would fetch information and data. We would need to make sure if the project did utilise the museum Wi-Fi, it would use as little data as viable.  
