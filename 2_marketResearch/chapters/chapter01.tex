Based on the requirements of the stakeholder, the app needs to be able to,

\begin{itemize}
    \item navigate the user to an exhibit or room through the use of augmented reality
    \item to display navigational routes in real time
    \item calculate the shortest route to the user specified location 
    \item work transferably in other museums/commercial spaces
    \item contain accessability features such as magnified text and inverted colours for example
\end{itemize}

After consulting and meeting with our stakeholders and potential users of the proposed application, we were able to gather a better understanding of what the apparent need was, in the relative market regarding commercial spaces and museums in particular. Out of the 21 responses we received, 15 potential users had admitted to visiting museums at least once a month. This shows that there is some level of frequency in their visits and that there is something that can be offered to this group of people. As our proposed application intends for the use of navigation around museums and commercial spaces, an important question that had to be asked was "do you find yourself using the maps in the museum more than once?" - a very reassuring 100\% of visitors had agreed that they did in fact refer to the maps around the museum more than once, some respondents going on to say that they referred to it over 10 times. However, these maps are not free. In most museums, including the Natural History Museum and the Science Museum in London, require a fee of £1 in order to have access to the paper maps they have available. This shows that there is an evident need for an accessible tool other than the maps around the museum in order to assist visitors' navigation around the museum. In addition to this, 18 of the respondents had agreed they would much rather prefer using their phone as a means of navigation rather than the paper maps that are currently available to assist in their navigation around the museum. These responses that we received first-hand were very reassuring for us as developers, as it brings to light an evident need for these visitors to have access to an improved navigation solution.