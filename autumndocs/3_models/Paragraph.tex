\documentclass{article}
\usepackage[utf8]{inputenc}

\title{Summary}
\author{Hamza Sheikh and Gabriel Sampaio Diogo}
\date{22-10-2018}

\begin{document}

\maketitle

\subsection*{Service Model}
In order to grasp the service model around the use cases exist from. We have to first exclaim that the following cases are born out of one important principle; convenience. The \textbf{'lost'} use case, for example, comes from the fact that the user could be lost for whatever reason. What we would provide through this service would be the quickest and most \textbf{convenient} solution to finding their destination. Whether that be the exit, a cafe or a particular exhibition. Another use case;
\textbf{'exploration'}, would become more convenient with the museum, and all its exhibitions (along with brief descriptions) at your fingertips (instead of on a wall-map or a pillar-map).
\subsubsection*{Model around two cases (The lost and the exploring) }
The lost-case and exploration-case has a virtually linear-stream of logic, and is as follows:
\begin{enumerate}
    \item The user enters within the radius of an environment modelled by the service. In this case, a museum.
    \item The user’s location is picked up once they give use permission to (in this case, it would typically be when the user opens the app). 
    \item The user then picks their destination.
    \item That location is then taken and parsed through a function containing an algorithm that calculates the most convenient route between the user’s real-time location and their desired destination.
    \item The user is then displayed the route, and directed towards their destination via their camera.
    \item Once done, the user is given curated suggestions on possible places they can go.
\end{enumerate}
\end{document}
