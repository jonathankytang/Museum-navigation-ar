% The design should be based on a UML use case, sequence and activity analysis covering the interest of the various stakeholders who would be involved in the use and deployment of your concept. The key interactions employed by users of your system should be identified.

{
\setstretch{1.5}
\section{Unified Modeling Language (UML)}
The implementation of the use case diagram outlines different scenarios in which users would function the application. (Figure~\ref{fig:Use Case Diagram}). It represents the functional behaviour of the system in terms of the goals (as defined in the stakeholder requirements) that can be fulfilled by the system.

UML was implemented to further support, and refine the designing phase of the software development through an activity diagram. (Figure~\ref{fig:Activity Model Diagram}). It was designed to model the work flow of the system; this was essential as these diagrams are easily comprehensible for both analysts, and stakeholders. By producing these models, there is a clear understanding of what the application does, and enables the visualisation of the application for the future.

\section{Service Model}
The following cases are based on convenience for the user. The \textbf{lost} use case comes from the user that could be lost for whatever reason. The service provided would be the quickest solution to finding their destination whether that be the exit or a particular exhibition. The \textbf{exploration} case, would be more convenient with the museum, and all its exhibitions will be at the user's fingertips (instead of existing museum navigation options).

\subsection*{Model around two cases}
Both cases have a linear-stream of logic:

\begin{enumerate}
    \item User enters within the radius of an environment (museum) modelled by the platform.
    \item User’s location is picked up once they give permission to.
    \item User selects their destination.
    \item That location is then taken, and passed through an algorithm calculating the quickest route between the user’s real-time location, and their destination.
    \item User is displayed the route, and directed towards their destination via their camera.
    \item User is given curated suggestions on possible places they can go.
\end{enumerate}

}