% This bring summarise and bring together all the previous section into a specification that should be fully expanded in the appendix with the following points, in providing what is known as the System Requirements Specification (SRS). This collects various information that you have previously agreed and worked on such as the UML diagrams.

% SRS contents:
% 1. Purpose
% 2. Scope
% 3. System Overview
% 4. References
% 5. Definitions
% 6. Use Cases
% 7. Functional requirements
% 8. Non-functional requirements.

{
\setstretch{1.5}
The SRS was designed to show the structured collection of the requirements of the system along with its operational environments, and external interfaces \cite{IEEE24765}. The specification was written according to the ISO standard for systems and software engineering \cite{IEEE29148}.

The purpose, scope, and system overview serve to provide an outline of the overall platform. Use cases diagrams from chapter four are fully described in the specification.

The functional requirements comprise of the system behaviour, the functions and features (\textit{what} the system should do). It considers the key features such as the user navigation and its relative implications.

Whereas the non-functional requirements place constraints on \textit{how} the system should do it. This considers the usability of the application, describing aspects such as performance, user actions, and safety.

\section*{Constraints, Assumptions and Dependencies}
\begin{enumerate}
    \item \textbf{Internet Connection}: The application would not be able to query mapping services or have access to exhibit information otherwise.
    \item \textbf{Android}: Users of this application are Android device users that requires assistance in museum navigation. Devices that support basic dependencies of the application is expected for proper user experience.
\end{enumerate}

}