\documentclass{article}
\usepackage[utf8]{inputenc}

\title{Functional specification}
\author{akhar003 }

\date{November 2018}

\begin{document}
\maketitle

\section{Introduction}
What is a functional specification?

Functional specification describes the important technical requirements for a system. It also includes the procedures in which the requirements have been met.

In this section, we are breaking down the functional specifications of our idea and how it should respond to a given task:

1.	The system should allow the user to enter their current location and their destination, this is key as it allows the app to calculate the route.
2.	The system should be able to calculate and work out the shortest and closest route to a given exhibition in short amount of time.
3.	The system should be able to display the route on the user’s smart phone in good quality and allow them to follow the route to their destination. The route would be displayed using a thick blue arrow which goes all the way to their destination, so all the users have to do is follow the arrow. For users with disability the app would have a voice guide.
4.	The system should ask the user for their rating of the museum and if they would suggest the exhibition to anyone else.
5.	Lastly, the system should be able to display different users reviews and past visits on the system, the reason for this is that it would use this information in the future to suggest to new users where they should visit and what museum would be suitable to them. The way the data for this would be collect is that once the user has reached their destination. The app would display a rating screening including a feedback option where users can give their honest opinion on their experience. 
\end{document}
