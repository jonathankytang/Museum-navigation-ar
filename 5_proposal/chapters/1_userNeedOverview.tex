The main concept for the project revolves around the user of augmented reality on smartphones. Augmented reality (AR) is the superimposing of a computer-generated image onto a user's view of the real world \cite{oxforddict}. This technology first came about in the 1960s but has recently gained consumer and wide-spread industry attention after the use of it in Snapchat filters and the 2016 game \textit{Pokémon Go} for example. There have been many times where people get lost in unfamiliar spaces such as a museum, where they are immersed by culture and their sense of direction. This project aims to tackle this issue by allowing the users to restore their orientation by having a mobile platform to route users to their destination, using AR. The platform will use the device's camera to work out its surrounding, and will produce a highlighted line on the screen to their destination in real time.\\ 

This concept has other various applications to other similar scenarios such as finding products in a supermarket, books in a library, or even valuable items that people own that can emit an electronic signal for it to be tracked down. Further, the concept could also use machine learning (ML) in identifying user's traits in places visited in a museum for example in order to give personalised recommendations at other similar exhibitions.