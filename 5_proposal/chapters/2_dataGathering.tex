% You should identify the stakeholders of any software you are developing and/or the motivation behind any research. You should describe how you gathered relevant data, summarised and analysed that data, and list project requirements based on that analysis.

The main stakeholders are museum visitors and staff. After consulting with them, and potential users of the proposed application, we were able to gain a better understanding of what the apparent need was in the relative market regarding museums. Out of the 21 responses we received, 15 potential users admitted to visiting museums at least once a month. This shows that there is some level of frequency in their visits, and that there is something that can be offered to this group of people.\\

Since our concept principally considers the user of navigation in museums, when users were asked, "do you find yourself using the maps in the museum more than once?"- a very reassuring 100\% of visitors had agreed that they did in fact refer to the maps around the museum more than once, some respondents going on to say that they referred to it over 10 times. However, these maps are not free; in most museums, including the Natural History Museum and the Science Museum in London, require a fee of £1 in order to have access to the paper maps.\\

This shows that there is an evident need for an accessible tool other than the maps around the museum in order to assist visitors' navigation around the museum. 18 of the respondents had agreed they would much rather prefer using their phone to navigate rather than the paper maps that are currently available to assist in their navigation around the museum. These responses that we received first-hand were very reassuring for us as developers, as it brings to light an evident need for these visitors to have access to an improved navigation solution.\\

Based on the stakeholder research, the project requirements are, 

\begin{itemize}
    \item navigate the user to an  through the use of augmented reality
    \item to display navigational routes in real time
    \item calculate the shortest route to the user specified location 
    \item work transferrably in other museums/commercial spaces
    \item contain accessibility features such as magnified text and inverted colours for example
\end{itemize}