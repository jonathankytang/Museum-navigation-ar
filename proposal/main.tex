\documentclass[12pt]{report}
\usepackage[utf8]{inputenc}
\usepackage[english]{babel}
\usepackage{graphicx}
\graphicspath{{images/}}
\usepackage[a4paper,top=25mm,bottom=30mm,width=135mm]{geometry}
\usepackage{lscape}
\usepackage{afterpage}
\setcounter{tocdepth}{1}
\newcommand\tab[1][0.75cm]{\hspace*{#1}}

\newcommand\blankpage{%
    \null
    \thispagestyle{empty}%
    \addtocounter{page}{-1}%
    \newpage}

\usepackage{fancyhdr}
\pagestyle{fancy}
\fancyhead{}
\fancyhead[C]{\leftmark}
\renewcommand{\headrulewidth}{0.4pt}

\usepackage[]{algorithm2e}

% !BIB TS-program = biber
% !BIB program = biber
\usepackage{listings}
\usepackage[nottoc]{tocbibind}
\addcontentsline{toc}{chapter}{Bibliography}
% \usepackage{biblatex}
% \addbibresource{references.bib}

\begin{document}
\begin{titlepage}
    \newgeometry{width=150mm,top=40mm,bottom=40mm}
    \begin{center}
         \vspace*{1cm}
        Department of Computing\\
        Goldsmiths, University of London\\
        
        \vspace*{3.75cm}
        
        \textbf{\Large Augmented Reality Navigation System for Commercial Spaces}\\
        \vspace*{0.25cm}           
        Software Projects\\  
    
        \vspace*{2cm}
        by\\
        \vspace*{0.25cm}    
        \textbf{Jonathan Tang, Hamza Sheikh, Hardik Ramesh, Arif Kharoti,\\}
        \textbf{Gabriel Sampaio Da Silva Diogo, Nicholas Orford-Williams}\\
        \vspace*{0.1cm}    
        Group 14

        \vspace{2cm}
        
        Autumn 2018
        \vfill
        
        Submitted in partial fulfillment for the degree of\\
        \textit{Bachelor of Science} in \textit{Computer Science}
        
        \vspace{1.5cm}
        
    \end{center}
\end{titlepage}
\afterpage{\blankpage}
\thispagestyle{plain}

\pagenumbering{roman}
\newgeometry{top=25mm,bottom=30mm,width=135mm}
\begin{center}    
	\large
    \textbf{Word Count}\\
    1219\\
    computed by \texttt{TeXcount}
\end{center}

\newgeometry{top=25mm,bottom=30mm,width=135mm}
\begin{center}    
    \large
    \textbf{Abstract}\\
\end{center}
Frustration and confusion are common emotions that are apparent at large shopping centres. After analysing recent studies, it is evident that shopping centres have a huge role to play in the overall retail experience. In order to provide greater value to both consumers and retailers, retail settings are being challenged to become smarter. One approach that is becoming increasingly recognised is mobile augmented reality (MAR) apps. Many consumers have difficulties in locating the store which satisfies their needs. In this research, we endeavour to outline the market requirement of developing an application that allows for smart retail and describing how additional value is created to customers as well as benefiting retailers. It is proposed that the application will implement a 3D model of various shopping centres, featuring navigation functionality to assist users in finding their desired store.\\

\tableofcontents
\afterpage{\blankpage}

\chapter{Market Research}
\pagenumbering{arabic}
\section{Current Solutions}

\section{Competitors}


\section{Studies on shoppers behvaiour}

\section{Retail experience}



\section{Footfall in Shopping Centres}
Lorem ipsum dolor sit amet, consectetur adipiscing elit, sed do eiusmod tempor incididunt ut labore et dolore magna aliqua. Eleifend quam adipiscing vitae proin sagittis nisl. Facilisis magna etiam tempor orci eu lobortis elementum nibh tellus. Massa tincidunt nunc pulvinar sapien. Malesuada nunc vel risus commodo viverra. Rhoncus urna neque viverra justo nec ultrices dui sapien. Id velit ut tortor pretium viverra suspendisse potenti. Aliquet enim tortor at auctor urna nunc id. Nulla porttitor massa id neque aliquam vestibulum morbi. Mauris pharetra et ultrices neque ornare aenean euismod elementum. Aenean vel elit scelerisque mauris pellentesque pulvinar pellentesque habitant morbi. Tempus egestas sed sed risus pretium quam vulputate dignissim. Leo integer malesuada nunc vel risus. Nunc sed velit dignissim sodales ut eu. Commodo odio aenean sed adipiscing diam donec adipiscing tristique. Id aliquet lectus proin nibh nisl. Augue eget arcu dictum varius duis at consectetur lorem. Dictum fusce ut placerat orci nulla pellentesque.

Leo vel fringilla est ullamcorper eget nulla facilisi etiam. In mollis nunc sed id semper risus. Ullamcorper eget nulla facilisi etiam dignissim diam. Ipsum suspendisse ultrices gravida dictum. Amet nulla facilisi morbi tempus iaculis urna id volutpat lacus. Volutpat odio facilisis mauris sit amet massa vitae tortor. Enim diam vulputate ut pharetra. Eu mi bibendum neque egestas. Sit amet nisl suscipit adipiscing. Eu non diam phasellus vestibulum lorem sed risus. Mauris pharetra et ultrices neque ornare aenean euismod elementum nisi. Ut enim blandit volutpat maecenas volutpat blandit aliquam etiam. Urna molestie at elementum eu facilisis sed odio. Purus semper eget duis at.

Sit amet commodo nulla facilisi nullam vehicula ipsum. Nibh sed pulvinar proin gravida. Vitae elementum curabitur vitae nunc sed. In iaculis nunc sed augue lacus viverra vitae congue eu. Luctus venenatis lectus magna fringilla urna porttitor rhoncus dolor purus. Urna nec tincidunt praesent semper feugiat nibh. Iaculis urna id volutpat lacus laoreet non. Pellentesque nec nam aliquam sem et tortor consequat id. Commodo odio aenean sed adipiscing. Nulla pharetra diam sit amet nisl suscipit adipiscing. Sed odio morbi quis commodo odio. Ipsum dolor sit amet consectetur adipiscing elit. Faucibus scelerisque eleifend donec pretium vulputate sapien. Vestibulum sed arcu non odio. Porttitor eget dolor morbi non arcu risus quis varius. Mi proin sed libero enim sed faucibus turpis. Aliquet eget sit amet tellus. Viverra vitae congue eu consequat ac felis donec et odio. Pulvinar elementum integer enim neque volutpat. Sit amet aliquam id diam maecenas ultricies mi eget.

Eros donec ac odio tempor orci dapibus. Amet mattis vulputate enim nulla aliquet porttitor lacus luctus accumsan. Nec tincidunt praesent semper feugiat nibh sed pulvinar proin. Justo eget magna fermentum iaculis eu. Sit amet purus gravida quis. Tempus egestas sed sed risus. Vitae aliquet nec ullamcorper sit amet risus nullam. Nisi scelerisque eu ultrices vitae auctor eu augue ut. At ultrices mi tempus imperdiet nulla malesuada pellentesque elit. Molestie a iaculis at erat pellentesque adipiscing. Venenatis urna cursus eget nunc scelerisque viverra mauris in. Volutpat lacus laoreet non curabitur gravida. Elit ullamcorper dignissim cras tincidunt lobortis. A condimentum vitae sapien pellentesque habitant morbi. Arcu risus quis varius quam quisque id diam vel. In nisl nisi scelerisque eu ultrices vitae auctor.

Donec ac odio tempor orci dapibus ultrices in. At tempor commodo ullamcorper a lacus vestibulum sed arcu non. Ultrices sagittis orci a scelerisque purus. Eleifend quam adipiscing vitae proin sagittis nisl. Netus et malesuada fames ac turpis egestas integer. Tempor orci dapibus ultrices in. Sit amet nulla facilisi morbi tempus iaculis urna id. Mi bibendum neque egestas congue quisque. Turpis nunc eget lorem dolor. Quis viverra nibh cras pulvinar mattis. Posuere lorem ipsum dolor sit amet.
\section{Regulations \& Standards}



% \chapter{Evaluation}

% \appendix
% \chapter{Sample chapter}

\bibliographystyle{unsrt}
\bibliography{references}

\end{document}

