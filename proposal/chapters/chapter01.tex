\section{Current Solutions \& Competitors}
The market of indoor museum navigation has become more competitive in recent years with more solutions being submitted due to a growth in indoor navigational research. Most current solutions on the market cater very well for a basic navigation of large public spaces, but will fail to display an even proportion of navigational and interactive content with well-presented data. Through the use of augmented reality, the concept can provide an interactive navigation solution for museums and exhibitions.\\
 
Since most museums and galleries use a portable audio guide, user experiences can be vastly improved by the use of a phone. Currently only a few solutions can be found; the Orpheo group \cite{orpheo} provide a unique app for each place meaning that their solution is somewhat cumbersome to regular museum users who would wish to have a hassle-free setup process. As we hope to appeal to museums and by virtue of this, museum-goers,having one app whereby the user can simply walk into a museum or exhibition and be greeted with relevant information to be a vital differentiating factor.\cite{microsoft} Even though the intention is to create one app for every museum and exhibition for the project, each client would have a large input on the content of their institution within the app.\\

If, as a museum, wanted a solution for navigation, due to the low number of museum-specific competitors, would choose to use a standard indoor mapping software. \cite{engadget} However, while there are many options out there from Google and Mapspeople \cite{mapspeople} who set out to provide this, they lack important exigences that are very imperative for museums like heavily integrated augmented reality, intelligent tour guiding from your location, and virtual reality to take a scene from the museum, for instance, and place the user to the artefact's original time and place.

\section{Studies on Museum Visitors' Behaviour and the Retail Experience}
It was proposed by Flavia Sparacino \cite{sparacino} the categorisation of museum visitors into three main categories: 

\begin{enumerate}
\item the greedy visitor who wants to know and see as much as possible;
\item the selective visitor who spends time on artefacts that represent certain concepts only and neglects the others; and (iii) the busy visitor who prefers strolling through the museum in order to get a general idea of the exhibition without spending much time on exhibits. Based on this, excluding  
\item the majority of museum visitors will find it beneficial to have a supportive application on their hand-held devices to assist in their navigation around the museum.
\end{enumerate}
Navigation is a fundamental exercise in browsing various exhibitions - awareness of navigation support available allows for both the museum management and visitors to benefit.\\

The visitor experience is continued through the museum shop. Museum shops provide an opportunity for cultural institutions to increase their revenue \cite{murphy}. However, public spending on museums has declined by 13\% in real terms over a decade, from £829 million in 2007 to £720 million in 2017 \cite{pickford}. Not only does this emphasise the importance of the satisfaction of museum visitors, but it also presents an undeniable truth, being that museum managers and retailers will want to
retain a long-term relationship with their customers. Wayfinding difficulties are psychological barriers in the built environment. They have an averse effect on our emotional state, causing stress, anger and humiliation \cite{Ecology}. To some people, complex buildings may become completely inaccessible as these visitors cannot find their way within acceptable level of risk-taking and energy investment \cite{Wayfinding}.\\ 

Evidently, there is a need for visitors to have access to support in navigating around the space, giving them ease of access to their areas of interest. One of the reasons for the decline mentioned is due to the vast amount of information that is readily available online, if a potential museum visitor would like to know more about an exhibit of an event in history, their answers would lie on their hand-held devices. A visitor survey commissioned by V\&A Digital Media and Learning departments and
conducted by Fusion/Frankly Webb and Green, has lots of fascinating details about visitors' uptake of personally-owned mobile devices, the way they use them in different contexts and their attitudes to digital content and services in the Museum such as WiFi. For example, a surprising result that challenges preconceptions of what the typical museum visitors actually do with their mobile devices: "Have you ever used your smartphone at a gallery or a cultural site to enhance your visit for any
reason? (Sample size: 258)" This question yielded a surprising result of only 40\% of interviewees answering 'No' whilst the remaining 60\% answering 'Yes' \cite{Lewis}. This allows for consideration of additional features of the application such as an informative implementation that could have the answers to any questions a visitor may have about a specific exhibit.

\section{Regulations \& Standards}
The field of augmented reality (AR) and virtual reality (VR) is currently not heavily regulated in the UK owing to the emergence of this new technology in recent times along with others such as blockchain and machine learning. There are certain areas such as data protection, intellectual property (IP), and security that need to be strongly factored in and considered during the development lifecycle. It should be noted that AR will involve collecting an extensive amount of data per user such as names, age and email address, but also appearance, real time location, and their interaction with other users.  Since the concept of the project relies on the user's camera, accelerometer, and location data on their smartphone, ensuring that this data cannot be obtained unlawfully and fits the scope of the Data Protection Act (1998) along with the EU General Data Protection Regulation (GDPR) is of most importance.\cite{ITProPortal}\\

Based on large VR companies such as Oculus, these obligations are addressed by the form of a privacy policy in order to detail how data is collected, used and if it is shared with third parties. Since GDPR presents many pitfalls for developers, it is critical these regulatory issues are addressed before the completion of the product and not after. Penalties for non-compliance can be up to \pounds 17 million or 4\% of annual turnover. \cite{eversheds}\\

Another regulatory standard is the intellectual property (IP) of the app. The source code and object code that serves as the underlying foundation of the app will be be original and qualify for copyright protection. Since computer software is usually excluded from patentability in the UK, any ideas that uses AR producing a technical effect, and its associated hardware can be protected by patents. Based on our competitors, it is important that we do not infringe on their patents owned by third parties. Equally, if the concept makes new technical developments in the field relating to AR, then it should be considered whether it would be eligible for patent protection.\\

Given that the AR experience is built using a database of information about the real world, the database can be protected by copyright. The concept could take on a machine learning viewpoint by recognising third party logos captured on the user's camera. This could cause an infringement claim since AR could be replicating, replacing trademark or copyright works, or distorting the logo.\\
