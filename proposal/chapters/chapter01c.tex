\section{Regulations \& Standards}
The field of augmented reality (AR) and virtual reality (VR) is currently not heavily regulated in the UK owing to the emergence of this new technology in recent times along with others such as blockchain and machine learning. There are certain areas such as data protection, intellectual property (IP), and security that need to be strongly factored in and considered during the development lifecycle. It should be noted that AR will involve collecting an extensive amount of data per user such as names, age and email address, but also appearance, real time location, and their interaction with other users.  Since the concept of the project relies on the user's camera, accelerometer, and location data on their smartphone, ensuring that this data cannot be obtained unlawfully and fits the scope of the Data Protection Act (1998) along with the EU General Data Protection Regulation (GDPR) is of most importance.\cite{ITProPortal}\\

Based on large VR companies such as Oculus, these obligations are addressed by the form of a privacy policy in order to detail how data is collected, used and if it is shared with third parties. Since GDPR presents many pitfalls for developers, it is critical these regulatory issues are addressed before the completion of the product and not after. Penalties for non-compliance can be up to \pounds 17 million or 4\% of annual turnover. \cite{eversheds}\\

Another regulatory standard is the intellectual property (IP) of the app. The source code and object code that serves as the underlying foundation of the app will be be original and qualify for copyright protection. Since computer software is usually excluded from patentability in the UK, any ideas that uses AR producing a technical effect, and its associated hardware can be protected by patents. Based on our competitors, it is important that we do not infringe on their patents owned by third parties. Equally, if the concept makes new technical developments in the field relating to AR, then it should be considered whether it would be eligible for patent protection.\\

Given that the AR experience is built using a database of information about the real world, the database can be protected by copyright. The concept could take on a machine learning viewpoint by recognising third party logos captured on the user's camera. This could cause an infringement claim since AR could be replicating, replacing trademark or copyright works, or distorting the logo.\\