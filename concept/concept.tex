% --------------------------------------------------------------
% This is all preamble stuff that you don't have to worry about.
% Head down to where it says "Start here"
% --------------------------------------------------------------
 
\documentclass[12pt]{article}
 
\usepackage[margin=1in]{geometry} 
\usepackage{amsmath,amsthm,amssymb}
 
\newcommand{\N}{\mathbb{N}}
\newcommand{\Z}{\mathbb{Z}}
 
\newenvironment{theorem}[2][Theorem]{\begin{trivlist}
\item[\hskip \labelsep {\bfseries #1}\hskip \labelsep {\bfseries #2.}]}{\end{trivlist}}
\newenvironment{lemma}[2][Lemma]{\begin{trivlist}
\item[\hskip \labelsep {\bfseries #1}\hskip \labelsep {\bfseries #2.}]}{\end{trivlist}}
\newenvironment{exercise}[2][Exercise]{\begin{trivlist}
\item[\hskip \labelsep {\bfseries #1}\hskip \labelsep {\bfseries #2.}]}{\end{trivlist}}
\newenvironment{reflection}[2][Reflection]{\begin{trivlist}
\item[\hskip \labelsep {\bfseries #1}\hskip \labelsep {\bfseries #2.}]}{\end{trivlist}}
\newenvironment{proposition}[2][Proposition]{\begin{trivlist}
\item[\hskip \labelsep {\bfseries #1}\hskip \labelsep {\bfseries #2.}]}{\end{trivlist}}
\newenvironment{corollary}[2][Corollary]{\begin{trivlist}
\item[\hskip \labelsep {\bfseries #1}\hskip \labelsep {\bfseries #2.}]}{\end{trivlist}}
 
\begin{document}
 
% --------------------------------------------------------------
%                         Start here
% --------------------------------------------------------------
 
%\renewcommand{\qedsymbol}{\filledbox}
 
\title{\textbf{Virtual Shopping Navigator}\\Project Concept - Group 14}%replace X with the appropriate number
\author{Jonathan Tang, Hamza Sheikh, Hardik Ramesh, Nicholas Orford-Williams,\\
Gabriel Sampaio Da Silva Diogo, Arif Kharoti\\ %replace with your name
\\
IS52018C: Software Projects 2018/19} %if necessary, replace with your course title
 
\maketitle

The main concept for the project revolves around the use of augmented reality on smartphones. There have been many a time where shoppers are at a large shopping centre, and they are trying to find a particular store but are unsure of the direction. The project aims to tackle this issue by allowing the user to input their destination store in the shopping centre, and for the app to calculate the distance between the user and the store. The app will use the camera to work out the user's
surrounding and will produce a highlighted line on the phone screen to the destination store.\\

We are looking at various avenues to build the app based on the current skillset of the group; Python has many computer vision libraries and Apple have recently developed the ARKit for developers to use on iOS devices. There are also various datasets we would need to gather such as store information and to create various 3D models of particular shopping centres.\\

This concept has potentially various applications to other scenarios such as finding products in a supermarket, books in a library, or even valuable items that people own that can emit an electronic signal, for it to be tracked down.

% --------------------------------------------------------------
%     You don't have to mess with anything below this line.
% --------------------------------------------------------------
 
\end{document}
