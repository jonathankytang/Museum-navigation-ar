%This bring summarise and bring together all the previous section into a specification that should be fully expanded in the appendix with the following points, in providing what is known as the System Requirements Specification (SRS). This collects various information that you have previously agreed and worked on such as the UML diagrams.

% SRS contents:
% 1. Purpose
% 2. Scope
% 3. System Overview
% 4. References
% 5. Definitions
% 6. Use Cases
% 7. Functional requirements
% 8. Non-functional requirements.

% \section{Difference between functional Requirement(FR) and non-functional requirement(NFR)}

% \subsection{What is functional requirement?}
% It's a requirement that described what system should do. In other word, it would express the behaviours of function of the system when certain conditions are made for example making an Ar system to help navigate around the exhibit.

% \subsection{What is non-functional requirement?}
% It's a requirement that described how the system should perform a certain function. It will describe the behaviour of the system and limits.

%\section{Difference between functional requirements (FR) and non-functional requirements (NFR)}
%As the application is a tool to support navigation around musuems, through the implementation of AR, there are two main types of requirements that the Systems Requirements Specification comprises: Functional and Non-Functional requirements. The functional requirements specify \textit{what} the system should do. Whereas, the non-functional requirements specify \textit{how} the system should perform a function.\\

%An example of how these requirements would differ: the system being able to successfully calculate the best route to their desired destination would be a functional requirement as it is defines a function of the software. However, the graphical interface and the design of the application is a non-functional requirement as it specifies criteria which can be used to judge the operation of a system.

%\subsection{What are functional requirements?}
%A requirement that describes what the system should do. In other word, it should express the behaviours of function of the system when certain conditions are made for example making an AR system to help navigate around the exhibit.

%\subsection{What are non-functional requirements?}
%It's a requirement that described how the system should perform a certain function. It will describe the behaviour of the system and limits.

\subsection{The System Requirement Specification (SRS)}

The SRS was designed to describe all data, functional and behavioral requirements of the platform. Covering the overall description of the system to be implemented, including the purpose and scope of the system, use cases and scenarios, functional and non-functional requirements, as well as constraints on the system.

\begin{enumerate}
    \item The functional requirements are:
    \begin{enumerate}
        \item Needs to be able to navigate the user from point A to B.
        \item The app should be able to display navigational routes in real time.
        \item It should be able to calculate the quickest route.
        \item A 3D line will be superimposed that navigates the user to their destination
        \item The user’s camera can recognise artwork/objects, displaying further information about the piece.
        \item when user arrives to their destination, the app will give recommendation based on their current route.
    \end{enumerate}
    \item The non-functional requirements are 
    \begin{enumerate}
        \item Performance -  The app should have quick response whenever a user wants to find more information about an art piece or whenever they search up a location. As well as that, it should not slow down their phone to avoid negative impacts.
        \item Usability - The app would be layout simple, there would be some colours used for appropriate purpose. The language used on the app would be easy to understand for the users. Having done research based on the user preference, it was discover that users tend to dislike having too much displayed on their menu screen all out once. Therefore, the app would take the user step by step to make it more user friendly.
        \item Data Usage - the app would need internet connection in order for it to work out the real time distance and the length of the final destination.
    \end{enumerate}
\end{enumerate}

\begin{enumerate}
    \item \textbf{Constraints, Assumptions and Dependencies:
    \begin{enumerate}
        \item \textbf{Internet Connection}: The application would not be able to query mapping services or have access to exhibit information otherwise.
        \item \textbf{Android}: Users of this application are any Android device user that requires assistance in museum navigation. An android device that can support basic dependencies of the application is expected for proper user experience.