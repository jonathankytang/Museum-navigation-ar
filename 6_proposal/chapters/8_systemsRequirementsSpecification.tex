%This bring summarise and bring together all the previous section into a specification that should be fully expanded in the appendix with the following points, in providing what is known as the System Requirements Specification (SRS). This collects various information that you have previously agreed and worked on such as the UML diagrams.

% SRS contents:
% 1. Purpose
% 2. Scope
% 3. System Overview
% 4. References
% 5. Definitions
% 6. Use Cases
% 7. Functional requirements
% 8. Non-functional requirements.

The SRS was designed to describe all data, functional and behavioral requirements of the platform. Covering the overall description of the system to be implemented, including the purpose and scope of the system, use cases and scenarios, functional and non-functional requirements, as well as constraints on the system.\\

The functional requirements comprise the behaviour of the system, the functions and features (\textit{what} the system should do), whereas the nonfunctional requirements place constraints on \textit{how} the system should do it.

\begin{enumerate}
    \item The functional requirements are:
    \begin{enumerate}
        \item Needs to be able to navigate the user from point A to B.
        \item The app should be able to display navigational routes in real-time.
        \item It should be able to calculate the quickest route.
        \item A 3-D line should be superimposed to display navigation route to the user's destination.
        \item Camera recognition of artwork/exhibits, displaying further information about the exhibit.
        \item When user arrives at destination, the app should give recommendation based on their current route.
    \end{enumerate}
    \item The non-functional requirements are:
    \begin{enumerate}
        \item \textbf{Performance}: The app should respond quickly to user input, e.g user wants to find more information about an art piece or whenever they search up a location. The system should not require extensive CPU usage, it should not slow the device down inconveniencing the user.
        \item \textbf{Usability}: The app should have a simple layout, with appropriate colour used in appropriate contexts. The language used on the app should be easy to understand for the users. Having done research based on the user preference, it was discovered that users dislike too much text on their menu screen.
        \item \textbf{Data Usage}: Data usage should be kept to a minimum, only querying the relevant information (user location and exhibit information). Also, the app would require internet connection in order to calculate the real-time distance of the final destination from the user's current location.
    \end{enumerate}
\end{enumerate}

\begin{enumerate}
    \item \textbf{Constraints, Assumptions and Dependencies}:
    \begin{enumerate}
        \item \textbf{Internet Connection}: The application would not be able to query mapping services or have access to exhibit information otherwise.
        \item \textbf{Android}: Users of this application are any Android device user that requires assistance in museum navigation. An android device that can support basic dependencies of the application is expected for proper user experience.
    \end{enumerate}
\end{enumerate}