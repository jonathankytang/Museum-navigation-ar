%This bring summarise and bring together all the previous section into a specification that should be fully expanded in the appendix with the following points, in providing what is known as the System Requirements Specification (SRS). This collects various information that you have previously agreed and worked on such as the UML diagrams.

% SRS contents:
% 1. Purpose
% 2. Scope
% 3. System Overview
% 4. References
% 5. Definitions
% 6. Use Cases
% 7. Functional requirements
% 8. Non-functional requirements.

% \section{Difference between functional Requirement(FR) and non-functional requirement(NFR)}

% \subsection{What is functional requirement?}
% It's a requirement that described what system should do. In other word, it would express the behaviours of function of the system when certain conditions are made for example making an Ar system to help navigate around the exhibit.

% \subsection{What is non-functional requirement?}
% It's a requirement that described how the system should perform a certain function. It will describe the behaviour of the system and limits.

\section{Differences between Functional and Non-Functional Requirements}
Non-functional requirements are more critical than FR because even if we missed one or two FR, the system will work properly whereas if we missed any NFR their will be problem where system can became useless. 
\begin{enumerate}
    \item The functional requirements are:
    \begin{enumerate}
        \item Needs to be able to navigate the user from point A to B.
        \item The app should be able to display navigational routes in real time.
        \item It should be able to calculate the quickest route.
        \item A 3D line will be superimposed that navigates the user to their destination
        \item The user’s camera can recognise artwork/objects, displaying further information about the piece.
        \item when user arrives to their destination, the app will give recommendation based on their current route.
    \end{enumerate}
    \item The non-functional requirements are 
    \begin{enumerate}
        \item Performance -  The app should have quick response whenever a user wants to find more information about an art piece or whenever they search up a location. As well as that, it should not slow down their phone to avoid negative impacts.
        \item Usability - The app would be layout simple, there would be some colours used for appropriate purpose. The language used on the app would be easy to understand for the users. Having done research based on the user preference, it was discover that users tend to dislike having too much displayed on their menu screen all out once. Therefore, the app would take the user step by step to make it more user friendly.
        \item Data Usage - the app would need internet connection in order for it to work out the real time distance and the length of the final destination.
    \end{enumerate}
\end{enumerate}

