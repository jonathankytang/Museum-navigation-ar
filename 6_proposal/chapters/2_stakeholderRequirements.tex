% You should identify the stakeholders of any software you are developing and reason that they have an interest in your concept.

The main stakeholders are museum visitors and staff. After consulting with them, and potential users of the proposed application, a better understanding was available, concerning the apparent need that was in the relative market regarding museums. Out of the 21 responses received, 15 potential users admitted to visiting museums at least once a month. Showing some level of frequency in their visits, and that something can be offered to this group of people.\\

Since the concept principally considers the use of navigation in museums, when users were asked, "do you find yourself using the maps in the museum more than once?"- 100\% of visitors agreed that they referred to the maps around the museum multiple times, and some respondents over 10 times. However, these maps are not free; in most museums, including the Natural History Museum and the Science Museum in London, require a fee of £1.\\

18 of the respondents agreed they preferred using their phone to navigate rather than the paper maps. Outlining a clear need for an accessible tool other than the maps around the museum.\\

Based on the stakeholder research, the project requirements are, 

\begin{itemize}
    \item navigate the user to a museum through the use of augmented reality
    \item to display navigational routes in real time
    \item calculate the shortest route to the user specified location 
    \item work transferrably in other museums/commercial spaces
    \item contain accessibility features such as magnified text and inverted colours for example
\end{itemize}

Another key stakeholder are museum staff as they are instrumental to any on-the-ground assistance in terms of navigation. Furthermore, the application should endeavour to make it easier for museum staff to assist vistors.\\

The stakeholder requirements of museum staff are,

\begin{itemize}
    \item Exhibit an effective and easy-to-use design. 
    \item Be economic and effective in its use of data, as most data would be sourced from the museum Wi-Fi. 
    \item Written content and other media to be within control of the museum.
\end{itemize}

During the field research, museum-floor staff and receptionists were also consulted. The staff approached had all received a navigational inquiry, either from themselves or visitors. Although positive responses were received several concerns were cited,\\

\begin{itemize}
    \item Battery performance
    \item Data usage
    \item Ease of use
\end{itemize}