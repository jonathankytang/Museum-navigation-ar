% You should detail any issues of privacy, data protection or intellectual property rights that may arise, and how you will manage them. You should confirm that you will not be working with minors or vulnerable adults.

AR is currently not heavily regulated in the UK owing to the emergence of this new technology. It should be noted that AR will involve collecting extensive amounts of data per user such as names and emails, but also real time location, and interactions with other users. Within the scope of this project, we will not be working with minors and vulnerable adults. Since the concept of the project relies on the user's camera, accelerometer, and GPS on the user's device, ensuring this data cannot be obtained unlawfully, fitting the scope of the Data Protection Act (1998), and GDPR is of most importance.\cite{ITProPortal}\\

Based on large VR companies such as Oculus, these obligations are addressed by the form of a privacy policy, to detail how data is collected, used and if it is shared with third parties. It is critical these regulatory issues are addressed before the completion of the product and not after.\\

Another regulatory standard is the IP of the software. The source code that serves as the underlying foundation of the platform will be be original and qualify for copyright protection. Since computer software is usually excluded from patentability in the UK, any ideas that uses AR producing a technical effect, and its associated hardware can be protected by patents. Based on our competitors, it is important that we do not infringe on their patents owned by third parties.\\

Equally, if the concept makes new technical developments in the AR field, there should be consideration whether it would be eligible for patent protection. The project could take on a machine learning viewpoint by recognising artworks captured on the user's camera. This could cause an infringement claim since AR could be replicating, replacing trademark or copyright works, or distorting the artwork.