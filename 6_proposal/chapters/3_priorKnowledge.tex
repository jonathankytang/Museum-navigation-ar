% You should describe how you gathered relevant information for credible sources, summarised and analysed that data, and how that information altered the proposed concept. 
% Computer Science: you should explain the computer science problems presented by your project, satisfying the programme learning outcome “Apply computational thinking to the design and implementation of moderately complex computing systems”.

% Whilst conceptualising the project idea, there is huge importance in knowing the solutions and methodologies are already in place to tackle the apparent technological problem, in our case, navigation and specifically in museums.

\section{Current Solutions \& Competitors}
The market of indoor museum navigation has become more competitive in recent years with more solutions being submitted due to a growth in indoor navigational research. Most current solutions on the market cater very well for a basic navigation of large public spaces, but will fail to display an even proportion of navigational and interactive content with well-presented data. Through the use of augmented reality, the concept can provide an interactive navigation solution for museums and exhibitions.\\
 
Since most museums and galleries use a portable audio guide, user experiences can be vastly improved by the use of a phone. Currently only a few solutions can be found; the Orpheo group \cite{orpheo} provide a unique app for each place meaning that their solution is somewhat cumbersome to regular museum users who would wish to have a hassle-free setup process. As we hope to appeal to museums and by virtue of this, museum-goers,having one app whereby the user can simply walk into a museum or exhibition and be greeted with relevant information to be a vital differentiating factor \cite{microsoft}.\\

If a museum wanted a solution for navigation, due to the low number of museum-specific competitors, would choose to use a standard indoor mapping software \cite{engadget}. However, while there are many options out there from Google and Mapspeople \cite{mapspeople} who set out to provide this, they lack important exigencies that are imperative for museums like heavily integrated AR, intelligent tour guiding from your location, and virtual reality to take a scene from the museum, for instance, and place the user to the artefact's original time and place.