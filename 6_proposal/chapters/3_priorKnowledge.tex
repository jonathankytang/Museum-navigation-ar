% You should describe how you gathered relevant information for credible sources, summarised and analysed that data, and how that information altered the proposed concept. 
% Computer Science: you should explain the computer science problems presented by your project, satisfying the programme learning outcome “Apply computational thinking to the design and implementation of moderately complex computing systems”.

Whilst conceptualising the project idea, there is huge importance in knowing the solutions and methodologies are already in place to tackle the apparent technological problem, in our case, navigation and specifically in museums. It is also important to understand the stakeholders and the interest they have in the development of the application.

\section{Current Solutions \& Competitors}
The market of indoor museum navigation has become more competitive in recent years with more solutions being submitted due to a growth in indoor navigational research. Most current solutions on the market cater very well for a basic navigation of large public spaces, but will fail to display an even proportion of navigational and interactive content with well-presented data. Through the use of augmented reality, the concept can provide an interactive navigation solution for museums and exhibitions.\\
 
Since most museums and galleries use a portable audio guide, user experiences can be vastly improved by the use of a phone. Currently only a few solutions can be found; the Orpheo group \cite{orpheo} provide a unique app for each place meaning that their solution is somewhat cumbersome to regular museum users who would wish to have a hassle-free setup process. As we hope to appeal to museums and by virtue of this, museum-goers,having one app whereby the user can simply walk into a museum or exhibition and be greeted with relevant information to be a vital differentiating factor\cite{microsoft}.\\

If a museum wanted a solution for navigation, due to the low number of museum-specific competitors, would choose to use a standard indoor mapping software. \cite{engadget} However, while there are many options out there from Google and Mapspeople \cite{mapspeople} who set out to provide this, they lack important exigencies that are imperative for museums like heavily integrated AR, intelligent tour guiding from your location, and virtual reality to take a scene from the museum, for instance, and place the user to the artefact's original time and place.

\section{Studies on Museum Visitors' Behaviour and the Retail Experience}
It was proposed by Flavia Sparacino \cite{sparacino} the categorisation of museum visitors into three main categories: 

\begin{enumerate}
\item the greedy visitor who wants to know and see as much as possible;
\item the selective visitor who spends time on artefacts that represent certain concepts only and neglects the others;
\item the busy visitor who prefers strolling through the museum in order to get a general idea of the exhibition without spending much time on exhibits.
\end{enumerate}

Based on this, excluding the busy visitors, museum visitors will find it beneficial to have a supportive application on their mobile devices to assist in their navigation around the museum.\\

Public spending on museums has declined by 13\% in real terms over a decade, from £829 million in 2007 to £720 million in 2017 \cite{pickford}.\\

One of the reasons for the decline mentioned is due to the vast amount of information that is readily available online, if a museum visitor would like to know more about an exhibit of an event in history, they already have the answers on their phones. A visitor survey commissioned by V\&A Digital Media and Learning departments and conducted by Fusion/Frankly Webb and Green, contains fascinating details about the way in which visitors use their mobiles phones in different contexts, their attitudes to digital content and services in the Museum such as WiFi. For example, a surprising result that challenges preconceptions of what the typical museum visitors actually do with their mobile devices: "Have you ever used your smartphone at a gallery or a cultural site to enhance your visit for any reason? (Sample size: 258)" This question yielded a surprising result of only 40\% of interviewees answering 'No' whilst the remaining 60\% answering 'Yes' \cite{Lewis}. This allows for consideration of additional features of the application such as an informative implementation that details additional information about a specific exhibit.