% You should describe how you gathered relevant information for credible sources, summarised and analysed that data, and how that information altered the proposed concept. 
% Computer Science: you should explain the computer science problems presented by your project, satisfying the programme learning outcome “Apply computational thinking to the design and implementation of moderately complex computing systems”.

\section{Current Solutions \& Competitors}
The market of indoor museum navigation has become more competitive in recent years with more solutions being submitted due to a growth in indoor navigational research. Most current solutions on the market cater very well for a basic navigation of large public spaces, but will fail to display an even proportion of navigational and interactive content with well-presented data.\\

Since most museums and galleries use a portable audio guide, user experiences can be vastly improved by the use of a phone. Currently only a few solutions can be found; the Orpheo group \cite{orpheo} provide a unique app for each place meaning that their solution is somewhat cumbersome to regular museum users who would wish to have a hassle-free setup process. As the aim of this concept is to appeal to museums and by virtue of this, museum-goers, having an application whereby the user can simply walk into a museum or exhibition and be greeted with relevant information is vital in comparison \cite{microsoft}.\\

If a museum wanted a solution for navigation, due to the low number of museum-specific competitors, would choose to use a standard indoor mapping software \cite{engadget}. However, while there are many options out there from Google and Mapspeople \cite{mapspeople} who set out to provide this, they lack important exigencies that are imperative for museums like heavily integrated AR, intelligent tour guiding from your location, and virtual reality to take a scene from the museum, for instance, and place the user to the artefact's original time and place.\\

From a technological point of view, an apparent problem in the solutions that museums implement today, would be their paper maps which do not process real-time locations. AR allows for real-time data processing, picking up the user's current location and displaying the best possible route for the user to take through their device's camera. One of the huge benefits of implementing AR is that it is a very unique approach to today's navigation solutions, whilst also allowing for user's to create their own content enabling more opportunities to interact with the application.