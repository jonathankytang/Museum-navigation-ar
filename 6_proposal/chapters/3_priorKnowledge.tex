% You should describe how you gathered relevant information for credible sources, summarised and analysed that data, and how that information altered the proposed concept. 
% Computer Science: you should explain the computer science problems presented by your project, satisfying the programme learning outcome “Apply computational thinking to the design and implementation of moderately complex computing systems”.

\section{Current Solutions \& Competitors}
The market of indoor museum navigation has grown recently with more solutions being submitted due to the growth in indoor navigational research. Most solutions cater well for a basic navigation of large public spaces, but fail to display an even proportion of navigational, and interactive content with well-presented data.

Since most museums use portable audio guides, user experiences can be vastly improved by mobile devices. Currently only a few solutions can be found; the Orpheo group \cite{orpheo} provide a unique app for each place; their solution is cumbersome to regular museumgoers who wish to have a hassle-free setup. As the aim is to appeal to museums, having an application whereby the user can walk into a museum, and greeted with relevant information is vital in comparison \cite{microsoft}.

If museums wanted a solution for navigation, due to the low number of museum-specific competitors, would choose to use a standard indoor mapping software \cite{engadget}. However, while there are many options out there from Google and Mapspeople \cite{mapspeople} who try to provide this, they lack exigencies that are imperative for museums like heavily-integrated AR, and intelligent tour guiding from your location.

From a technological viewpoint, a problem in the solutions that museums implement today, would be their paper maps not processing real-time locations. AR allows for real-time data processing, picking up the user's current location, and displaying the best route for the user to take through their device's camera. A benefit of implementing AR is the unique approach to today's navigation solutions, whilst allowing users to create their own content, enabling more opportunities to interact with the application.