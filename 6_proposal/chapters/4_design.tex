% The design should be based on a UML use case, sequence and activity analysis covering the interest of the various stakeholders who would be involved in the use and deployment of your concept. The key interactions employed by users of your system should be identified.

\section{Importance of Design}
Having a design process allows for more efficiency, and transparency when coming to design the application. It overcomes the risk of referring back to the drawing board when developing the application, setting in stone the main features, and functionality of the application.

\section{Unified Modeling Language}
An effective design strategies was carried out through the implementation of the UML, a powerful standard for creating various specifications the software system.\\

Our implementation of a use case diagram outlined the different scenarios in which a user would function the application. (Figure~\ref{fig:Use Case Diagram}). UML was implemented was to further support, and refine the designing phase of the software development through an activity diagram. (Figure~\ref{fig:Activity Model Diagram}).\\

The use case diagram represents the functional behaviour of the system in terms of goals (as defined in the stakeholder requirements) that can be fulfilled by the system. The activity diagram was designed to model the work flow of the system. One main reason that the activity diagram was essential was that these diagrams are normally easily comprehensible for both analysts, and stakeholders. By producing these models, we were able to have a clear understanding of what the application does, and enabled us to visualise the application for the future.

\section{Service Model}
The following cases are born out of one important principle, convenience. The \textbf{lost} use case, for example, comes from the user that could be lost for whatever reason. The service we would provide would be the quickest and most convenient solution to finding their destination, whether that be the exit or a particular exhibition. The \textbf{exploration} case, would be more convenient with the museum, and all its exhibitions will be at the user's fingertips (instead of existing museum navigation options e.g. wall-maps or paper maps).

\subsection*{Model around two cases}
Both cases have a linear-stream of logic:

\begin{enumerate}
    \item The user enters within the radius of an environment (museum) modelled by the service.
    \item The user’s location is picked up once they give use permission to.
    \item The user picks their destination.
    \item That location is then taken, and passed through an algorithm calculating the quickest route between the user’s real-time location, and their destination.
    \item The user is then displayed the route, and directed towards their destination via their camera.
    \item The user is given curated suggestions on possible places they can go.
\end{enumerate}