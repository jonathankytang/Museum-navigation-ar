% How you intend to test and evaluate your software during and after development. It may be useful to specify individual test cases.
% TDD

\subsection*{Test Driven Development (TDD)}

TDD is the main developmental process of choice, both during and after development because of the following qualities. 

\subsubsection{1. Tangible results that can reviewed with efficacy}
As a test based method, it gives the creators the ability to prove what does and does not work asynchronously - if the project were to be developed with stand-up as the central process, this core quality of quick and effective development would be lost. As a result of its allowance for effective review, there is increased efficiency to adapt the application to the results. For example, the application requires a guiding graphic - TDD makes it easy to review whether a 3-D line in reality, is the best case solution or whether an alternative graphic is better such as a directional arrow.

\subsubsection{2. Simplicity of implementation}
As a relatively technically complex application, TDD helps break down each process effectively. For example, when choosing the best route - Does the algorithm place more weight on scalar distance? Or on obstacles? TDD makes the process of making this decision distinct and ultimately simple.

\subsubsection{3. Re-usability}
The project also only has a few purposes, making it optimal for iterative testing and prototyping. Sebsequently ensuring that the facets of the application can be developed to a high degree of quality through the implementation of TDD.