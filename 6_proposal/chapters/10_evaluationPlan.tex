% How you intend to test and evaluate your software during and after development. It may be useful to specify individual test cases.
% TDD

\subsection*{Test Driven Development (TDD)}

Is the developmental process of choice, both during and after creation because of the following. 
\subsubsection{1. Tangible results that can reviewed with efficacy}
As a test based method. It gives the creators the ability to prove what works, and what does not. It also privileges them with the ability to adapt. For example, the application requires a guiding graphic. TDD makes it easy to review whether a 3D line in reality, is the best case solution. Or whether an alternative, such as an arrow, is better.
\subsubsection{2. Simplicity of implementation}
The application as a concept. Is simple, but it's relatively difficult to manifest because of the technical complexities. To have a simple way to evaluate the application makes it easier to create. TDD here can help with the complexities of the route calculations. For example, when choosing the best route. Does the algorithm place more weight on scalar distance? Or on obstacles? TDD makes the process of making this decision distinct and therefore, simple.
\subsubsection{3. Re-usability}
In principle also. The project only has a few purposes. The ability to iterate over multiple prototypes and algorithms with TDD makes the process of fulfilling those few purposes simpler. It also means they can be achieved with a high degree of quality.