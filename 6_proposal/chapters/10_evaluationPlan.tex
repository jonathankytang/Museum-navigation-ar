% How you intend to test and evaluate your software during and after development. It may be useful to specify individual test cases.
% TDD

\subsection*{Test Driven Development (TDD)}

TDD is the main developmental process of choice, both during and after creation because of the following qualities. 

\subsubsection{1. Tangible results that can reviewed with efficacy}
As a test based method, it gives the creators the ability to prove what does and does not work asynchronously - if the project were to be developed with stand-up as the central process, this core quality, that gives way for quick and effective development would be lost, and because of its allowance for effective review, it makes it easy to adapt the application to the results. For example, the application requires a guiding graphic. TDD makes it easy to review whether a 3D line in reality, is the best case solution. Or whether an alternative, such as an arrow, is better.

\subsubsection{2. Simplicity of implementation}
As a relatively technically complex application, TDD helps break down each process effectively. For example, when choosing the best route. Does the algorithm place more weight on scalar distance? Or on obstacles? TDD makes the process of making this decision distinct and therefore, simple.

\subsubsection{3. Re-usability}
The project also only has a few purposes. Making it optimal for iterative testing and prototyping. Meaning that TDD also makes sure that the facets of the application can be created with a high degree of quality.