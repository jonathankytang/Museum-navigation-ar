% Summarise your proposal, including the key points from the previous sections.

{
\setstretch{1.5}
The concept is to build a solution to indoor museum navigation within its sector. Issues became transparent through market research, which defined the user requirements. Upon discovering the inadequacies of current solutions (chiefly failing in succinct user navigation with museum exhibitions) through the implementation of augmented reality the project will be able to offer real-time 3-D mapping.

UML models were drawn up using stakeholder requirements to define user and activity behaviours. Research was conducted around AR libraries to find the appropriate packages for the project. ARCore stood out as it can accurately anchor virtual objects better than Vulforia and ARKit. A storyboard, and three UI/UX prototypes were constructed to show stakeholders samples of the platform, then combining all effective features into one. This provided an understanding of user behaviour concomitant with usability.

From prototyping, the main functional elements were outlined. It was decided the platform will adopt the MVC architecture to aid modelling, and questions from the user stories were addressed. System requirements specification was composed to summarise, outlining features and behaviour of the system. During the system life-cycle, TDD will be used to test system components.

Ethical auditing helped to adhere to regulations, and standards in software development. Due to the wide scope of the project, Agile methodology will be used, allowing for ease of innovation in relation to features.

}