% Having validated the proposed solution with users and answered any open technical or feasibility questions, attribute specific technologies to the functional architecture and present this as a technical architecture. Justify your choice of technologies with reasoned arguments for rejecting or retaining alterative technologies.

\section{Means of Software Development}

\subsection*{SDKs}
Google's \textbf{ARCore} kit gives us the ability to apply the AR element of the application without having to spend time pre-defining AR methods. It has distinct advantages over Apple's ARKit as ARCore can detect horizontal surfaces that is similar to motion tracking, and can accurately anchor virtual objects. \cite{newgenapps}

\subsection*{Platform \& Languages}
The app will be developed on Android since ARCore only works on that platform. Java is imperative to the project since android development is only possible in this language.

\subsection*{IDE}
\underline{Android Studio} is the IDE utilised in the project because it involves a number of relevant exclusive packages. Other IDEs, requires them to be pre-defined, and therefore takes out valuable time from application development.

\subsection*{Architectural Pattern}
Our application fits under the MVC pattern perfectly be it that the following are true.
\begin{itemize}
    \item Model: Data provided by the user (e.g. geolocational data)
    \item View: Front-end interface (e.g. 3D line to location)
    \item Controller: Algorithms between the model \& view (e.g. route calculation)
\end{itemize}
The pattern's simplicity makes the most sensible one we can use.

\newpage

\section{Satisfying user-related questions from the user stories}
\subsection*{Questions}
\begin{enumerate}
    \item How will the navigation system get me from point A to point B? (Figure~\ref{fig:AtoB})\\\\
    In order for user to get from one point to another, it will use route calculation to calculate the quickest route.\\
    Route Calculations:
    \begin{itemize}
        \item Algorithms to request and process GPS signal.
        \item Algorithms to calculation quickest route when user enter their destination.
        \item Once calculated, show the result for user to start their journey.
    \end{itemize}
    
    \item How easy will it be to grasp the app?\\\\
    The layout would be simple and the basic map/guidance will work straightforwardly. Once the route has been calculated, a 3D line will be superimposed on the users screen.
    
    % \item When I open the app, what will I see? \\\\
    % When user opens the application, the first thing they will see is the main page. For example, sign in/guest users, and service utilities.
    
    \item Can the app be used without Internet? \\\\
    No, otherwise the app would not have access to the user's real time location, and would take up too much storage space on the user's device if it was used without.
\end{enumerate}

% \subsection*{User Stories}
% % There are three crucial user stories - they're shown below.
% The user stories satisfies our architectural pattern and can be easily realised with our technical infrastructure. \textbf{ARCore} and functions provided by \textbf{Android Studio} provide us with predefined technicalities such as superimposition, and real-time geo-spatial data which allow us to lay the foundations of all three processes.