% Describe the prototyping you did, and what you learned from this, particularly the low fidelity interaction prototypes, any technical prototypes to explore technical feasibility of the solution, and the functional digital prototypes that you used with users to validate that what you are developing meets the expectations of users.

\section{AR Libraries}
In order to identify libraries that are good for implementing AR on mobile devices, we divided this prototyping into three platforms to explore them, and built test applications to find out how they help with the project.

\subsection*{Vulforia (Unity/Android)}
Unity is a cross-platform game engine, used to test a simple AR camera prototype where the device's camera hovers an object/image, and displaying information about that object/image on the device. We used Vuforia, an SDK that enables recognition, and tracking of image targets, to build it. This library can be used for the exploration case in the use case model. Although, there is a limited amount of tools for locating user current location compared to Android.

\subsection*{ARKit (iOS)}
We built a similar prototype to Unity on Apple's ARKit using Swift, which was easy to learn. It was intuitive to implement AR features as there was detailed documentation but logging GPS data was harder compared to Android.

\subsection*{ARCore (Android)}
ARCore was used to create a simple 3D model showing on a mobile device when its camera targets a flat surface. Compared to iOS, it is easier to log GPS location, although connecting the user interface to the scripts was more challenging.

\section{UI/UX Designs}