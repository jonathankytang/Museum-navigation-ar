% Describe the prototyping you did, and what you learned from this, particularly the low fidelity interaction prototypes, any technical prototypes to explore technical feasibility of the solution, and the functional digital prototypes that you used with users to validate that what you are developing meets the expectations of users.

\section{Augmented Reality Libraries}
In order to identify libraries that are good for implementing AR on mobile devices, we divided this prototyping into three platforms to explore them, and built test applications to find out how they help with the project.

\subsection*{Vulforia (Unity/Android)}
Unity is a cross-platform game engine, used to test a simple AR camera prototype where the device's camera hovers an object/image, and displaying information about that object/image on the device. The application was built using Vuforia, an SDK that enables recognition, and tracking of image targets. This library can be used for the exploration case in the use case model. Although, there is a limited amount of tools for locating user current location compared to Android.

\subsection*{ARKit (iOS)}
A similar prototype to Unity was built on Apple's ARKit using Swift, which was easy to learn. It was intuitive to implement AR features as there was detailed documentation but logging GPS data was harder compared to Android.

\subsection*{ARCore (Android)}
ARCore was used to create a simple 3D model showing on a mobile device when its camera targets a flat surface. Compared to iOS, it is easier to log GPS location, although connecting the user interface to the scripts was more challenging.

\section{User Interface/User Experience Designs}
The project lends substantial importance to its user interface and experience. As it will be used from a wide cross section of technical ability, the aim for UI will be to make the app as simple, and easy to use as possible without having an impinging effect on any major service the end product will feature. This prerequisite was clearly outlined in the surveying of museum guests and staff alike. Our first mission was to determine what interfaces, and experiences current exists within the museum sector. Many museums did employ simple interfaces but due to their mass-manufacturing, their design felt unoptimized, slow and clunky, with simple barebones media not beyond text and images. Furthermore, this design would fail to deliver anything more complex than texts and images.\\
  
The approach to the UX/UI prototyping was to create a score of different complete interface mockups and exhibit them alongside existing solutions. Three team members independently drew up potential interfaces. These candidates were then put to stakeholders, and all received positive attributes were combined into one.