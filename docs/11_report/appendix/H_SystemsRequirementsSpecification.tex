\section{Purpose}
The main goal of this concept is to provide users a solution to indoor museum navigation through an exciting, and enjoyable experience using augmented reality. It includes users being lost, or searching for a specific location within the museum. It was discovered through field research that the concept would make life easier for users and the museums since it would allow easy access to the information based on exhibitions.

\section{Scope}
This project will include creating an AR application for people to get an enjoyable journey in the museum. The project will be completed by 29 April 2019. The AR application will include simple navigation system to direct various part of the museum. Getting information on the user screen using the user's camera, and explore various museum using the system. The platform will be developed on Android due to the high usage of Google's AR library, ARCore.

\section{System Overview}
The application will perform all the basic tasks to help users with their journey in the museum. Such as navigating from point A to B, getting the user back on track in case they are lost, allowing the user to view information based on camera recognition of an exhibit.

\section{References}
This specification should be read in conjunction with the following publications:\\
IEEE 24765-2017 - ISO/IEC/IEEE International Standard - Systems and software engineering--Vocabulary \cite{IEEE24765}\\
IEEE Std 29148-2011, ISO/IEC/IEEE International Standard - Systems and software engineering -- Life cycle processes --Requirements engineering \cite{IEEE29148} \\
IEEE Std 730-2014, IEEE Standard for Software Quality Assurance Processes \cite{IEEE730} \\
IEEE Std 24748-4-2016 - ISO/IEC/IEEE International Standard for Systems and Software Engineering -- Life Cycle Management -- Part 4: Systems Engineering Planning \cite{IEEE24748}

\section{Definitions}
\textbf{Activity:} A set of cohesive tasks of a process, which transforms inputs into outputs. [ISO/IEC/IEEE 12207:2008]\\
\newline
\textbf{Augmented reality:} A technology that superimposes a computer-generated image on a user's view of the real world, thus providing a composite view. \\
\newline
\textbf{Functional requirement:} A requirement that specifies a function that a system or system component must perform. [ISO/IEC/IEEE 24765:2010]\\
\newline
\textbf{Non-functional requirement:} The measurable criterion that identifies a quality attribute of a function or how well a functional requirement must be accomplished. A non-functional requirement is always an attribute of a functional requirement. [ISO/IEC/IEEE 730:2014]\\
\newline
\textbf{Performance:} Degree to which a system or component accomplishes its designated functions within given constraints, such as speed, accuracy, or memory usage. [ISO/IEC/IEEE 24765:2017]\\
\newline
\textbf{Stakeholder:} Individual or organization having a right, share, claim or interest in a system or in its possession of characteristics that meet their need and expectations. [ISO/IEC/IEEE 15288:2015]\\
\newline
\textbf{Usability:} Extent to which a system, product or service can be used by specified users to achieve specified goals with effectiveness, efficiency and satisfaction in a specified context of use. [ISO/IEC/IEEE 25064:2013]\\
\newline
\textbf{User:} Individual or group that interacts with a system or benefits from a system during its utilisation. [ISO/IEC 25010:2011]

