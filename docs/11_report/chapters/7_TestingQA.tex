\section{Testing conducted}
% Hardik Week 2
% not complete
This section present the testing and quality assurance of our project. During development our team created various testing rule to improve maintainability and limit bugs in our application. We explore various testing techniques used in Software Development Life Cycle(SDLC) and Test-Driven Development(TDD). All testing list of the system has be detailed in the Appendix G (Testing).

\note{following sections needs to relate back to our project}
\subsection{Unit Testing}
The unit testing was carried out in house by the team. The aim for this testing to evaluate the quality of the application and finding bugs to ensure the product works as expected. We have tested 14 different unit testing for our application and 5 of them test did not pass.

\subsection{Integration Testing}
It has been tested using incremental testing, taking on the 'top down' approach. As the nature of the application is an Augmented Reality navigation app, we can not tested AR \& Bluetooth functionality because it is still in progress to make the feature live.

\subsection{Performance and stress testing}
Performance testing examines stability, usability, reliability, responsiveness and speed of our application therefore testing was carried out to assess the performance of the application. Stress testing was carried out to check he upper limits of the application. As application will used in commercial space so our application need to make sure it does not crash during high volume of users. We have run and test our application in various devices to see the performance of each features in different mobile devices and also to check if the application crash during the installation process.

\subsection{Regression testing}
It is the process of testing changes to our application to make sure that the older iteration still works with the new features. Before releasing a new version of application, the old version are tested against the new version to make sure that all the old features still work with new program. During development, a two various branch was created in Git-Lab for route calculation. By testing and merging two branches will shows if previous iterations are still compatible with v.0.0.1 and v.0.0.2. During the testing, we pass this features in our application.

\subsection{User Acceptance Testing (UAT)}
The purpose behind UAT is to conform to the system being developed and ready for operational use according to the specified user requirements. We have tested seven various user acceptance where four of them can not be executed because the development team has not written the code. The only test we pass during testing was when the login my our application as guest. During the testing we found that user can not use camera functionality and select destination. This feature will be conducted in future with our next version.

\subsection{Beta Testing}
This is the final stage of the testing, where the application released to an external test group consisting of real users. It helps us to find if their is any bugs in our application before releasing to public. We carried out one UI function testing where user can input credentials, destination and exhibit information. At the end, the program pass all the users input and shows the correct outcome.

\section{Deployment}
Hamza Week 2

After the application is ready and successfully released, a notification will be sent to stakeholders and clients. All iterations of the system will be detailed in the changelog.

Notification of deployment procedure (Taken from Appendix F):

\begin{enumerate}
	\item Check all procedures and ensure everything is done.
	\item Email client for meeting.
	\item Present the client with the application.
	\item Sign off documents.
	\item Email client with application information.
	\item Release of project and approval of supervisor.
\end{enumerate}

To deliver a solution of quality, the project employs a set of practices of continuous automated integration and deployment throughout development. The set of practices (properly pipelined, automated and frequently executed) allows for the developers to recieve feedback quickly about the quality of the software. To facilitate continuous integration effort the project uses GitLab's in house features.

\textbf{Source Code Control (Management)}: The project has its own repository in GitLab so that the source code and supporting documents can be tracked, maintained, versioned, and audited.

\textbf{Build Automation}: Build Automation is handled using Android Studio. All separate software components of the project are Android projects. For example, the route calculations was considered as a separate project in the repository, and the AR was built in a separate Android project. This allows for a common build interface and specification of concise instructions for the components testing, building and upload to the project's distribution repository. This is the initial initial step for continuous integration.

\textbf{Unit Test Automation}: Android Studio has JUnit built-in which is an open source unit test framework. During development, unit testing becomes much easier to perform as Android Studio allows for this as a feature.

\textbf{Deployment Automation}: Using an online continuous integration and delivery platform such as CircleCI, it is possible to automate the deployment process through the use scripts. Google also provides an API to edit a PlayStore listing which will be used to upload the APK and publish it. This can be done directly through HTTP as it will have to be used in combination with CircleCI.

\section{Formative evaluation}
Everyone

\section{Functional requirements review}
% Arif Week 2
% Hamza Week 2
To verify and validate that the requirements have been met, as mentioned previously, the team had adopted a test-driven development (TDD) approach which ensured that the team was working towards a fully functional system. The system successfully navigates a user from one point to another. This was achievable by displaying navigational routes in real-time and superimposing a 3-D line with AR technology. The quickest route was calculated by implementing the A* algorithm and tested in regression testing during development. However, other requirements have not been yet due to discussions made by the team regarding further iterations of the application. For example, although developed in protoyping - the requirement of implementing camera recognition on artworks/exhibits and displaying further information about them was not met but was decided to be inlcuded in further iterations of the application. However considering the complexity of the application, this was a reasonable decision to make as the Minimal Viable Product agreed upon by the team was to deliver the foundational service of the system: AR-assisted navigation. Lastly, the requirement of having the system suggest recommendations based on the user's route was also not met. However, this was also decided by the team to not be included in the first release of the application; suggesting recommendations is a feature that can only be implemented with a database. Due to the lack of resources available at hand, having an account database would be too costly as it would require the use of a server. Stakeholder feedback encouraged the introduction of an account system, therefore implementation of an account system is possible in further development of the application.

\section{Non-Functional requirements review}
% Arif Week 2
% Hamza Week 2
As with the functional requirements, implementing a TDD approach greatly helped the team adhere to the requirements as much as possible, ensuring the foundational requirements are met. For example, the performance of the system should not be slow - after a new feature was added to the system, the developers would then test-run the app to make sure that the new feature did not affect the performance. Similarly the usability was also tested to ensure ease of use - going back to the stakeholders to gain feedback on the simplicity of the navigation of the system. Once the stakeholders were happy, it meant that the requirement was met. In terms of the data usage of the application on the user's device; the only real data being sent out by the device is the user location and internet connectivity. It was imperative the system requested the user for location permissions and the application is also able to function on Wi-Fi so this requirement was always kept in mind during development. In terms of safety, development has not reached a level of complexity where the user would need to concerned about hitting an immutable object in their path - however, further development will always house this concern and take it into consideration. Having avoided the use of Bluetooth for route calculations, the concern of security in the user's location being accessed by third-party users was invalidated. 