\section{Stakeholders}
% Nick Week 1
% changes not fixed
Stakeholders hold a very important position during a product's development; often influencing the outcome of a project by way of its scope, means of function and overall effectiveness and success. In foresight of this impact, we were keen to detect and appoint all appropriate audiences to this rank.

Fundamentally, the group at the centre is the End User; in near-to-every circumstance, if those at the bottom on any sort of hierarchy reject an idea or practice, those above will not benefit through its implementation. Furthermore, this is exacerbated as this project will largely change the End User’s experience of a museum which, if rejected, may lead to a decline in visits and, thereafter, in funding. To understand our potential impact, expectations from visitors and members of museum staff of the Science Museum and the Natural History Museum were noted. 

Though this project was the product of an academic environment, it is not too unlike the real world in that we had we had direction set by this project’s supervisor, Dr. Basil Elmasri  and standards set by Dr. Nick Hine – the module leader. Given this project had to adhere to College's marking criteria, our supervisors responsibility was to ensure that we were at least doing so. Furthermore, the quality of work produced would be representative of them. 

As this project is framed around being a main point of interaction between museums and visitors, it could affect the experience and potentially having an adverse effect. Because of this, those in industry had a stake in this project. ‘Those in industry’ is two-fold: the technology sector and creators of exhibits or those who know about the representation thereof. It was a priority to get an experienced understanding so a line of communication with an company called fuse who have an incubator company, Pathfindr (navigate), who provide indoor navigation solutions, via the Managing Director, Matt Isherwood.

Seeking opinions from those in the business of being responsible for exhibits or creative works, a priority was to gauge this stakeholder via showing prototype designs and ideas. Getting the full scope of their opinion in relation to this project as our project will directly change how the end users interacts with the exhibit, leading to either hindering or bettering their experience. 

Finally, the team behind this project, have a large stake in its success. Not only that if there was not a quality report, but this new part of the technology sector is in its infancy so needs innovative ideas. Further to this, a ill digested report would act favorably towards the project sponsor, Dr Elmasri.

\section{Gathering}
% Nick Week 1
% not complete
\note{planning to merge this with the user consultation section}
In presenting our System Requirements Specification (SRS) to our stakeholders, we were able to determine via our surveying their approval. Along with (cite stakeholder survey response) Feedback was positive and we were further encouraged by staff describing the current system of charging the current system of charging for maps and their current app solution as inadequate and cited often having to help visitors having simple issues navigating. We made it our business to visualise as much as possible the user experience to ease its otherwise, seemingly complicated description. Constructing the diagram showing our use case made it easy to demonstrate the project and, furthermore, inspired, jointly with our user requirements, the first graphical prototypes took shape. Using Adobe XD, we quickly made a mockup of the final design which allowed for a clear commination twix us and the stakeholders. 

\section{System}
% Arif Week 1
% not complete - too similar to the proposal
The system requirements that it is an Android Mobile application, written in Android studios. using Java. The reason we decided to use Android was because Android is accessible on more devices and is able to greater number of audience. The SRS was designed to show the structured collection of the requirements of the system along with its operational environments, and external interfaces. The specification was written according to the ISO standard for systems and software engineering.
The purpose, scope, and system overview serve to provide an outline of the overall platform. Use cases diagrams from chapter four are fully described in the specification.
The functional requirements comprise of the system behaviour, the functions and features (what the system should do). It considers the key features such as the user navigation and its relative implications.
Whereas the non-functional requirements place constraints on how the system should do it. This considers the usability of the application, describing aspects such as performance, user actions, and safety.

A software requirement specification otherwise known as SRS is a description of a software system to be developed. SRS is used to layout the functional and non-functional requirements of a system. As well as that, a software requirement specification should include definitions of both system and user requirements, it should describe fully what the software will be.

Our system specification requirements clearly defines what type of software application we are going to be building. We used the SRS to know all the requirements for our application, which helped us in designing the software. System specification requirements creates a bond between the developers and clients, it allows them come to a conclusion about what should be implemented and how it should be implemented. Without a SRS it would be hard to know what needs/ doesn't need to be implement, also it creates a lot of mess and confusion between the clients and developers, and creates a mismatch of expectations. Which in the end is disadvantageous to both customer and developer.

\section*{Constraints, Assumptions and Dependencies}
\begin{enumerate}
    \item \textbf{Internet Connection}: The application would not be able to query mapping services or have access to exhibit information otherwise.
    \item \textbf{Android}: Users of this application are Android device users that requires assistance in museum navigation. Devices that support basic dependencies of the application is expected for proper user experience.
    \item \textbf{ARCore enabled device}: users device must be compatible with ARCore for them to have access to the application's AR navigation.
\end{enumerate}

\section{Functional}
% Arif Week 1
% not complete - too bullet pointy
The functional requirements comprise of the system behaviour, the functions and features (what the system should do). It considers the key features such as the user navigation and its relative implications.

The user should be able to use the application to navigate between the given current location and the required destination. The system should be able to display navigational routes in real-time from the users given location calculating the quickest possible route. A 3D line should be superimposed through augmented reality making use of the users camera to display the navigation to the user’s destination.
Camera recognition, detecting artwork/exhibits should display further information about the exhibit. When the user arrives at destination, the system should give a recommendation based on their route.

\section{Non-Functional}
% Arif Week 1
% not complete - too bullet pointy
The non-functional requirements place constraints on how the system should do it. This considers the usability of the application, describing aspects such as performance, user actions, and safety.
\begin{enumerate}
	\item \textbf{Performance}: The system should respond quickly to user input, e.g user wants to find more information about an art piece or whenever they search up a location. The system should not require extensive CPU usage, it should not slow the device down inconveniencing the user.
	\item \textbf{Usability}: The system should have a simple layout, with appropriate colour used in appropriate contexts. The language used on the app should be easy to understand for the users. Having done research based on the user preference, it was discovered that users dislike too much text on their menu screen.
	\item \textbf{Data Usage}: Data usage should be kept to a minimum, only querying the relevant information (user location and exhibit information). Also, the app would require internet connection in order to calculate the real- time distance of the final destination from the user’s current location.
	\item \textbf{Safety}: The system needs to have the ability to detect immutable objects obstructing the user’s path. This can reduce common user accidents when using a mobile phone whilst walking.
	\item \textbf{Security}: Ensuring that the device’s current location cannot be obtained by unauthorised third-party users is crucial in ensuing the security of using the platform.
\end{enumerate}
