\section{Stakeholders}
Nick Week 1
% changes not fixed
Stakeholders hold a very important position during a product's development; often influencing the outcome of a project by way of its scope, means of function and overall effectiveness and success. In foresight of this impact, we were keen to detect and appoint all appropriate audiences to this rank.

Fundamentally, the group at the centre is the \change{End User. In}{end users; in} In near-to-every circumstance, \note{not so keen on the hierarchy point here} if those at the bottom on any sort of hierarchy reject an idea or practice, those above will not benefit through its implementation. Furthermore, this is exacerbated as this project will largely change the \change{End User's}{end user's} experience of a museum which, \annote{if rejected by the that ilk}{rephrase this}, may lead to a decline in visits \remove{and, thereafter, in funding}. To understand our potential impact, we surveyed visitors and junior members of museum staff of the Science Museum and the Natural History Museum. 

Though this project was the product of an academic environment, it \change{isnt't}{is not} too unlike the real world in that we had we had direction set by this project’s supervisor, Dr. Basil Elmasri  and standards set by Dr. Nick Hine – the \remove{Group Project Module Leader}{module leader}. Given this project had to adhere to College's marking criteria, our supervisors responsibility was to ensure that we were at least doing so. Furthermore, the quality of work \remove{we would} produced would \remove{also} be representative of them. 

As this project is framed around being a main point of interaction between museums and visitors, it could affect the experience and potentially having an adverse effect. \annote{Because of this, we deemed it so, those in industry had a stake in this project. For us, 'those in industry' is two-fold: indoor navigation and creators of exhibits or those who know about the representation thereof.}{rephrase this} \annote{There has been a flurry of interest recently into indoor navigation across the technology sector. Given the excitement surrounding this,}{already mentioned in chapter 2} we were keen to get an experienced understanding. We were able to communicate with an incubator company, Pathfindr, via the Managing Director, Matt Isherwood. \note{what does pathfindr do} \remove{We see these experts as key stakeholders not only through our assassination, but as we are representatives for this growing chapter in computing.}

Seeking opinions from those in the business of being responsible for exhibits or creative works, we made it a priority of to gauge this stakeholder \annote{via online surveying}{change this to another interviewing method}. Getting the full scope of their opinion in relation to this project as our project will directly change how \change{the End User}{end users} interacts with the exhibit, leading to either hindering or bettering their experience. 

\annote{Finally, we, as the team behind this project, have a somewhat large stake in its success. Not only that if we didn't submit a quality report, but this new part of the technology sector may lose out on what we have, or would have, found.}{rephrase}
\note{need to mention who the project sponsor would be, or basil would be acting it basically}

\section{Gathering}
Nick Week 1

\note{planning to merge this with the user consultation section}
In presenting our System Requirements Specification (SRS) to our stakeholders, we were able to determine via our surveying their approval. Along with (cite stakeholder survey response) Feedback was positive and we were further encouraged by staff describing the current system of charging the current system of charging for maps and their current app solution as inadequate and cited often having to help visitors having simple issues navigating. We made it our business to visualise as much as possible the user experience to ease its otherwise, seemingly complicated description. Constructing the diagram showing our use case made it easy to demonstrate the project and, furthermore, inspired, jointly with our user requirements, the first graphical prototypes took shape. Using Adobe XD, we quickly made a mockup of the final design which allowed for a clear commination twix us and the stakeholders. 

\section{System}
Arif Week 1

\note{say what the system requirements are (i.e. describe what hardware/OS/middleware software should be run on.) and the justifications for that}
The SRS was designed to show the structured collection of the requirements of the system along with its operational environments, and external interfaces. The specification was written according to the ISO standard for systems and software engineering. The purpose, scope, and system overview serve to provide an outline of the overall platform. Use cases diagrams from chapter four are fully described in the specification.

% What is SRS?
A software requirement specification otherwise known as SRS is a description of a software system to be developed. SRS is used to layout the functional and non-functional requirements of a system. As well as that, a software requirement specification should include definitions of both system and user requirements, it should describe fully what the software will be. \note{These two paragraphs are basically the same, combine into one that makes sense.}

% {How we used it?}
Our system specification requirements clearly defines what type of software application we are going to be building. We used the SRS to know all the requirements for our application, which helped us in designing the software. \note{more detail needed}

% {why did we use SRS?}
\add{The} system specification requirements creates a bond between the \change{developers}{development team} and \change{clients,}{key stakeholders;} it allows them come to a conclusion about what \remove{should be implemented} and how it should be implemented. Without \change{a}{the} SRS it would be hard to know what needs doesn't need to be implement, also it creates a lot of \remove{mess and} confusion between \change{the clients and developers}{stakeholders}, and creates a mismatch of expectations. \remove{Which in the end is disadvantageous to both customer and developer.} \note{more reasoning needed here - look up online why the srs is good}

\section*{Constraints, Assumptions and Dependencies}
\begin{enumerate}
    \item \textbf{Internet Connection}: The application would not be able to query mapping services or have access to exhibit information otherwise.
    \item \textbf{Android}: Users of this application are Android device users that requires assistance in museum navigation. Devices that support basic dependencies of the application is expected for proper user experience.
    \item \textbf{ARCore enabled device}: users device must be compatible with ARCore for them to have access to the application's AR navigation.
\end{enumerate}

\section{Functional}
Arif Week 1

\note{moved this from the system requirements section}
\add{The functional requirements comprise of the system behaviour, the functions and features (what the system should do). It considers the key features such as the user navigation and its relative implications.}

\note{You need to go into more detail about each point, look at how you wrote the non-functional section, and check the essay plan}
\begin{enumerate}
	\item \textbf{}: Needs to be able to navigate the user from point A to B.
	\item \textbf{}: The system should be able to display navigational routes in real-time.
	\item \textbf{}: It should be able to calculate the quickest route.
	\item \textbf{}: A 3D line should be superimposed through augmented reality to display the navigation to the user’s destination.
	\item \textbf{}: Camera recognition on artwork/exhibits, displaying further information about the exhibit.
	\item \textbf{}: When user arrives at destination, the system should give a recommen- dation based on their route.
\end{enumerate}

\section{Non-Functional}
Arif Week 1

\note{moved this from the system requirements section}
\add{The non-functional requirements place constraints on how the system should do it. This considers the usability of the application, describing aspects such as performance, user actions, and safety.}
\begin{enumerate}
	\item \textbf{Performance}: The system should respond quickly to user input, e.g user wants to find more information about an art piece or whenever they search up a location. The system should not require extensive CPU usage, it should not slow the device down inconveniencing the user.
	\item \textbf{Usability}: The system should have a simple layout, with appropriate colour used in appropriate contexts. The language used on the app should be easy to understand for the users. Having done research based on the user preference, it was discovered that users dislike too much text on their menu screen.
	\item \textbf{Data Usage}: Data usage should be kept to a minimum, only querying the relevant information (user location and exhibit information). Also, the app would require internet connection in order to calculate the real- time distance of the final destination from the user’s current location.
	\item \textbf{Safety}: The system needs to have the ability to detect immutable objects obstructing the user’s path. This can reduce common user accidents when using a mobile phone whilst walking.
	\item \textbf{Security}: Ensuring that the device’s current location cannot be obtained by unauthorised third-party users is crucial in ensuing the security of using the platform.
\end{enumerate}
