\section{Summative evaluation}
Everyone

\section{Future developments}
Nick Week 2

By design, this project is geared towards to meeting the requirement of the museum sector, including public galleries. However, the main objective navigation, being so widely applicable, with adequate research there are many prudent areas of development. However, with the saturation of navigation and mapping software, this project should strive to stick to the indoor avenue. 

In a commercial context the most important future development would be retail navigation. As the product is based on finding the quickest route to a point in a museum, taking the project to this stage would not require any fundamental change. As in the museum concept, a retail version would again involve finding the best solution in terms of pure navigation and also proposing certain shops in-keeping with the user’s preferences. However, it should be made clear that based on our current model which involves studying the layout of a candidate area, a need to find a quicker method of digitising a space is desirable. 

\note{also the other things we outlined in the backlog for feature iterations}