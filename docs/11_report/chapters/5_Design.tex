\section{Models}
Gabe week 1

\subsection{Use Case}


\subsection{Activity}


\section{User Interface}
Hamza Week 1

The project lends substantial importance to its user interface and experience. As it will be used by people with a wide range of technical ability, the aim will be to make the app as simple as possible without having an impinging effect on any service the end product will feature. This prerequisite was clearly outlined in the surveying of museum guests and staff alike. The first mission was determining what interfaces, and experiences currently exists within the museum sector. Many museums employed simple interfaces but due to their mass-manufacturing, their design felt unoptimised with simple bare-bones media not beyond text and images. Furthermore, this design would fail to deliver anything more complex than texts and images.
  
The approach to the UI/UX prototyping was to create different interface mock ups and exhibit them alongside existing solutions. An initial storyboard was drawn up and three potential interfaces (Figure~\ref{fig:prototype1}~\ref{fig:prototype2}~\ref{fig:prototype3}) were designed and shown to stakeholders. The feedback gained from the stakeholders was invaluable in the process as it allowed for the group to \change{understand the positive and negative attributes}{consider all aspects} of the prototypes. It was decided that a final version of the application's user interface would be designed, implementing all the positive attributes, and combining it into one (Figure~\ref{fig:finaloverview}), whilst also considering the negative attributes.

Figure here

% \subsubsection{Prototype 1}
% \begin{figure}[H]
%     \centering
%     \includegraphics[width=\textwidth]
%     {prototypes/ui/1.png}
%     \caption{Overview of UI Prototype 1}
%     \label{fig:prototype1}
% \end{figure}

% \subsubsection{Prototype 2}
% \begin{figure}[H]
%     \centering
%     \includegraphics[width=\textwidth]
%     {prototypes/ui/2.png}
%     \caption{Overview of UI Prototype 2}
%     \label{fig:prototype2}
% \end{figure}

% \subsubsection{Prototype 3}
% \begin{figure}[H]
%     \centering
%     \includegraphics[width=\textwidth]
%     {prototypes/ui/3.png}
%     \caption{Overview of UI Prototype 3}
%     \label{fig:prototype3}
% \end{figure}

\newpage

Figure here
% \subsubsection{Final Prototype}
% \begin{figure}[H]
%     \centering
%     \includegraphics[angle=90, width=\textwidth]
%     {prototypes/ui/final.png}
%     \caption{Overview of final UI prototype}
%     \label{fig:finaloverview}
% \end{figure}

\newpage


\section{Technical Architecture}
Gabe Week 1

\section{Accessibility}
Arif Week 1

Accessibility is about making \change{your application usable}{the product} to majority of \change{your audience}{end users}. \change{Our Application}{The application} provides accessibility as part of the service we are offering to our audience and a way to make our app more generally appealing. Some accessibility options would include features such as relating to mobility, colour perception, \add{and} literacy \remove{etc}.

\begin{enumerate}
	\item \textbf{Screen Reading}: \remove{Screen reading is one of features which we have used in our app.} Users tend to rely on a screen reader to help them interact with the app, \change{therefore our app includes}{including} UI elements, such as name, role, description, state and value. This feature would allow the user to use the app without any difficulties since the layout is simple and straightforward. 

	\item \textbf{Keyboard Accessibility}: This accessibility feature would allow users to interact with all UI elements within the application by keyboard only. It enables users to navigate through the app by using arrow keys. Allows the users to activate UI elements in the app by using Enter keys. Allows the users to enter their current and final location using the keyword. \note{This one is too bullet pointy - can you make it more prose like so it flows better?}

	\item \textbf{Scale}: \note{Don't use first person} We used scale to allow users to zoom and resize some elements, our main thought behind this was to help people with visual impairment, especially for images that include words. We also made sure to not start with a font size that is too small in general for many users, the reason behind this was because everyone’s vision deteriorates as they age; and we wanted to make sure our app is usable for users of any age. \remove{That's all that we have been able to get done so far, however}\add{In future iterations}, we are thinking of allowing differences in vision, the way we are doing this will be by providing different scaling options for our users in the app settings. This means they can change the font size for easier reading or even shrink or enlarge the UI as well for better vision. \note{talk about how scale affects the augmented reality part, and how to avoid motion sickness}

	\item \textbf{Vision}: The text and UI in \change{our}{the} app is designed to support high-contrast theme. \remove{We know that} \change{while}{Whilst} colour is important, it must not be the only channel of communicating information. For instance, users who are colour blind would not be able to distinguish some colour status indicators from their surroundings. Therefore \change{we have included}{,} other visual cues such as text \add{are included} in order to ensure the on the app information is accessible to everyone. 
    
\end{enumerate}

\section{User consultations}
Hamza Week 1

\note{Might convert this section to requirements gathering}
\remove{Consultation addressed potential} \add{End-}users and other key stakeholders such as an interested domain expert, Matt Isherwood, Managing Director at Fuse\textsuperscript{\textcopyright}, \add{were consulted during the design stage} to clarify \change{their}{user} needs and requirements. In addition to our \remove{online questionnaires and } online interactions with our stakeholders, the group had also conducted field research, and had face-to-face talks with potential users at \change{museums in London (e.g.} the Natural History Museum, Science Museum, \add{and} the V\&A museum \remove{)} to obtain \add{a} more nuanced input from key stakeholders. When prototyping the application, drawing and designing the user interface was not as simple as bringing ideas to life - it was imperative that our users were consulted when making these decisions. 

Upon consulting the users, it became apparently easier to actualize the designs put forward - acknowledging the fact that the users had all required a system that was relatively simple to navigate around. However, users were also consulted when it came to the technical aspect of the project. For example, deciding which medium would be best to track user location was one of the questions that arised in the early stages of development. \add{In that regard,} \remove{consulted} Mr Isherwood \add{was consulted} about this, and he \remove{had} brought to light the plethora of solutions that were readily available to begin development, primarily highlighting that the use of \change{bluetooth}{Bluetooth} beacons would be most efficient to implement given the scope of the project. \note{Why did Matt say Bluetooth was good?} \note{Add somewhere we spoke to some Fine Art students at the University of Oxford to give us perspective on user experience etc.}