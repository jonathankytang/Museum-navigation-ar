%-----------------------
% PREAMBLE
%-----------------------
\documentclass[12pt]{report}
\usepackage[utf8]{inputenc}
\usepackage[english]{babel}
\usepackage{appendix}
\usepackage{graphicx}
\graphicspath{{figures/}}
\usepackage[a4paper,top=20mm,bottom=30mm,width=140mm]{geometry}
\usepackage{pdflscape}
\usepackage{afterpage}
\usepackage{float}
\usepackage{subcaption}
% \usepackage{subfig}
\usepackage[
    backend=biber,
    style=ieee,
    urldate =comp
  ]{biblatex}
 
 \addbibresource{references.bib}

\setcounter{tocdepth}{0}
\newcommand\tab[1][0.75cm]{\hspace*{#1}}

\newcommand\blankpage{%
    \null
    \thispagestyle{empty}%
    \addtocounter{page}{-1}%
    \newpage}

\usepackage{fancyhdr}
\pagestyle{fancy}
\fancyhead{}
\fancyhead[C]{\leftmark}
\renewcommand{\headrulewidth}{0.4pt}

\usepackage[]{algorithm2e} 
\usepackage[nottoc]{tocbibind}
\usepackage{url}
\usepackage[some]{background}
\usepackage[intoc, english]{nomencl}
\makenomenclature
% \renewcommand{\nomname}{List of Abbreviations}

% \setlength{\parindent}{4em}
% \setlength{\parskip}{1em}
% \renewcommand{\baselinestretch}{1.25}
\usepackage{setspace}
\usepackage{csquotes}

% \usepackage[some]{background}

% \SetBgScale{1}
% \SetBgContents{\parbox{10cm}{%
%       \Huge Draft:  \today\\[14cm]\rotatebox{180}{\Huge Draft:  \today}}}
%       \SetBgColor{gray}
%       \SetBgAngle{270}
%       \SetBgOpacity{0.2}

%-----------------------
% FRONT PAGE
%-----------------------
\begin{document}
\begin{titlepage}
    \newgeometry{width=150mm,top=40mm,bottom=40mm}
    % \BgThispage
    \begin{center}
        \vspace*{1cm}
        Department of Computing\\
        Goldsmiths, University of London\\

        \vspace*{3.25cm}

        \textbf{\LARGE Augmented Reality Navigation System\\}
        \vspace*{0.20cm}           
        \textbf{\LARGE for Commercial Spaces}\\
        \vspace*{0.55cm}           
        {\large Report}\\
        \vspace*{0.15cm}           

        \vspace*{2cm}
        by\\
        \vspace*{0.25cm}   

        \textbf{Arif Kharoti, Nicholas Orford-Williams, Hardik Ramesh,\\}
        \textbf{Gabriel Sampaio Da Silva Diogo, Hamza Sheikh, Jonathan Tang\\}
        \vspace*{0.1cm}    
        Software Projects – Group 14\\  

        \vspace{2cm}

        Spring 2019
        \vfill

        Submitted in partial fulfillment for the degree of\\
        \textit{Bachelor of Science} in \textit{Computer Science}

        \vspace{1.5cm}

    \end{center}
\end{titlepage}

\afterpage{\blankpage}

\thispagestyle{plain}
\pagenumbering{roman}

%-----------------------
% ABSTRACT
%-----------------------

\newgeometry{top=25mm,bottom=30mm,width=140mm}
\begin{center}        
    \large
    \textbf{Abstract}\\
\end{center}

The use of mobile augmented reality by consumers, and research in the field has become more prominent in the last decade. This has allowed for completely new approaches in solving current problems using this technology as there is a year-on-year increase on smartphone users across the world. 

This proposal presents the use of augmented reality in museum navigation on mobile devices. After conducting stakeholder research, there were clear issues presented by current solutions on the market through the form of paper maps. Augmented reality library research was conducted on various platforms to find the appropriate toolkit for the proposed system, and UI/UX prototyping prioritised key design aspects of the system. Following this, the technical architecture and user stories are defined through the model-view controller architectural pattern, along with technologies to be used during implementation. Methods and approaches to implementation are outlined, namely through the agile methodology along with consulting various testing methods.

\vspace*{1.5cm}
\begin{center}    
    \large
    \textbf{Word Count}\\
    xyz\\
    \normalsize computed by \texttt{TeXcount}
\end{center}

\vspace*{1.5cm}
\begin{center}    
    \large
    \textbf{Supervisor}\\
    \normalsize Dr. Basil Elmasri
\end{center}

\afterpage{\blankpage}

%-----------------------
% TOC & NOMENCLATURE
%-----------------------

\setcounter{tocdepth}{0}
\tableofcontents

\setcounter{tocdepth}{1}
\listoffigures

\nomenclature{AR}{Augmented Reality}
\nomenclature{GDPR}{General Data Protection Regulation}
\nomenclature{GPS}{Global Positioning System}
\nomenclature{IDE}{Integrated Development Environment}
\nomenclature{IP}{Intellectual Property}
\nomenclature{MVC}{Model-View Controller}
\nomenclature{SDK}{Software Development Kit}
\nomenclature{UI}{User Interface}
\nomenclature{UML}{Unified Modeling Language}
\nomenclature{UX}{User Experience}
\nomenclature{VR}{Virtual Reality}
\printnomenclature[1in]

\afterpage{\blankpage}

%-----------------------
% CHAPTERS
%-----------------------

\chapter{Concept Introduction \& User Needs}
\pagenumbering{arabic}
% % This should explain your project idea. You can presume that the reader has a basic level of technical knowledge but not specific knowledge of your particular area, so communicate your ideas clearly.

The main concept for this project revolves around the use of augmented reality (AR) navigation on smartphones. AR is the superimposing of a computer-generated image onto a user's view of the real world \cite{oxforddict}. This technology first came about in the 1960s \cite{InteractionDesign} but recently gained wide-spread consumer attention after the use of it on Snapchat filters \cite{Snapchat}, and the 2016 game \textit{Pokémon Go}. People get lost in unfamiliar spaces such as a museum, immersed by the culture around them, and their sense of direction. This project aims to tackle this issue by allowing users to restore their orientation by having an AR platform, routing users to their destination. The platform will use the device's camera to work out its surrounding, and produce a highlighted line on the screen to their destination in real time.\\

This concept has various applications to other scenarios such as finding products in a supermarket, or books in a library. Further, the concept could also use machine learning in identifying user traits in places visited in a museum in order to give personalised recommendations at other similar exhibitions.

\afterpage{\blankpage}

%-----------------------
% APPENDICES
%-----------------------
\begin{appendices}

\chapter{Systems Requirements Specification}
% \input{appendix/}

\end{appendices}

%-----------------------
% BIBLIOGRAPHY
%-----------------------
\addcontentsline{toc}{chapter}{Bibliography}
\printbibliography

\afterpage{\blankpage}
\end{document}
