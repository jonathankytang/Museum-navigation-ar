% How you intend to test and evaluate your software during and after development. It may be useful to specify individual test cases.
% TDD

{
\setstretch{1.5}
\section{Test Driven Development (TDD)}
TDD is the main developmental process of choice, both during and after development because of the following qualities. 

\subsection*{1. Tangible results that can reviewed with efficacy}
As a test based method, it gives creators the ability to prove what does, and does not work asynchronously - if the project were to be developed with daily stand-up meetings as the central process, this core quality of quick, and effective development would be lost. As a result of its allowance for effective review, there is an increased efficiency to adapt the application to the requirements. 

For example, the application requires a guiding graphic - TDD makes it easier to review whether a 3D line is the best case solution, or an alternative graphic is better such as a directional arrow. Crucially, this brings clarity to programmers when debugging code that is not their own, ensuring the level of quality assurance is the same standard for all features to be implemented.

\subsection*{2. Simplicity of implementation}
As a relatively technically complex application, TDD helps break down each process effectively. For example, when choosing the best route 

\begin{displayquote}
\textit{"Does the algorithm place more weight on scalar distance? Or on obstacles?"}
\end{displayquote}

TDD makes the process of making this decision distinct and ultimately simple. Compared to unit testing, and behaviour driven development (BDD), TDD allows for ease of maintainability, and adding new features to the system due to the large set of tests written to ensure that new features do not affect other parts of the current system \cite{codeutopia}. A simpler design can be achieved by creating modules that are low coupled, and have high cohesion with the system.

\subsection*{3. Re-usability}
The project only have a few purposes, making it optimal for iterative testing, and prototyping. Subsequently ensuring that the facets of the application can be developed to a high quality through the implementation of TDD.

}