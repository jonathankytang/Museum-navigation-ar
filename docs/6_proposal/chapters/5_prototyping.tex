% Describe the prototyping you did, and what you learned from this, particularly the low fidelity interaction prototypes, any technical prototypes to explore technical feasibility of the solution, and the functional digital prototypes that you used with users to validate that what you are developing meets the expectations of users.

{
\setstretch{1.5}
\section{Augmented Reality Libraries}
In order to identify libraries that are good for implementing AR on smartphones, prototyping was divided into three platforms to explore them, building test applications to discover how they assist with implementation.

\subsection*{Vuforia (Unity/Android)}
Unity is a cross-platform game engine, used to test a simple AR camera prototype where the device's camera hovers an image, and displays information about that image on the device (Figure~\ref{fig:vulforia}). The application was built using Vuforia, an SDK that enables recognition, and tracking of image targets. This library can be used for the exploration case in the use case model. Although, there is a lack of tools for locating user current location compared to Android.

\subsection*{ARKit (iOS)}
A similar prototype to Unity (Figure~\ref{fig:ARKit}) was built on Apple's ARKit using Swift \cite{applear}, which was easy to learn. It was intuitive to implement AR features as there was detailed documentation but logging GPS data was harder compared to Android.

\subsection*{ARCore (Android)}
ARCore was used to create a simple 3D model showing on a mobile device when its camera targeted flat surface (Figure~\ref{fig:ARCore}). Compared to iOS, it is easier to log GPS location (Figure~\ref{fig:Android sensors logging}), although connecting the user interface to the scripts was more challenging.

\section{User Interface/User Experience Designs}
The project lends substantial importance to its user interface and experience. As it will be used by people with a wide range of technical ability, the aim will be to make the app as simple as possible without having an impinging effect on any service the end product will feature. This prerequisite was clearly outlined in the surveying of museum guests and staff alike. The first mission was determining what interfaces, and experiences currently exists within the museum sector. Many museums employed simple interfaces but due to their mass-manufacturing, their design felt unoptimised with simple bare-bones media not beyond text and images. Furthermore, this design would fail to deliver anything more complex than texts and images.
  
The approach to the UI/UX prototyping was to create different interface mock ups and exhibit them alongside existing solutions. A storyboard and three potential interfaces (Figure~\ref{fig:prototype1}~\ref{fig:prototype2}~\ref{fig:prototype3}) were drawn up and put to stakeholders, implementing all the positive attributes (Appendix~\ref{Prototype Reviews}) were combined into one (Figure~\ref{fig:finaloverview}), and considering the negative attributes.

}
